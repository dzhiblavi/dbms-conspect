\subsection{Реляционная алгебра. Предназначение и свойства}
Базы данных нужно уметь не только проектировать, но и использовать. Существует несколько способов формулировать запросы. Первый из рассматриваемых -- \textit{реляционная алгебра}.

\paragraph{Мотивация}
Действительно, в базах данных можно не только хранить данные, но и делать выборки, изменять их каким-либо образом. Для этого вводится понятие запроса. При первом рассмотрении, запросы нужны как минимум для выполнения следующих действий:

\begin{itemize}
	\item Выборка данных: получить данные из базы, чтобы тем или иным способом обрабатывать их уже извне.
	\item Область действия обновлений: запросы позволят указывать область действия тех или иных операций, что крайне полезно. Например, к таким операциям относятся операции удаления или изменения данных: хочется указывать, на какие именно записи эти операции подействуют.
	\item Ограничения целостности: до сих пор было только два вида ограничений (ключи и внешние ключи). Некоторые базы данных позволяют создавать произвольные ограничения целостности, заданные на поддерживаемом языке. В рамках этих ограничений очень удобно пользоваться запросами.
	\item Ограничения доступа.
\end{itemize}

\begin{definition}
	\textit{Реляционная алгебра} -- алгебра над множеством всех отношений.
	% TODO(dzhiblavi) добавить ссылку %
\end{definition}

Далее будут определены некоторые из операций (которые по определению должны быть замкнуты над носителем), и ограничения, которые им соответствуют. В целом, реляционная алгебра -- императивный язык для работы с отношениями, который позволяет в явном виде, по действиям, описать, каким именно образом должен быть получен результат.

\paragraph{Примеры} Рассмотрим несколько простых примеров операций в рамках
реляционной алгебры.
\begin{itemize}
	\item Проекция отношения на множество атрибутов: $\pi_{A}(R)$;
	\item Естественное соединение $R_1 \bowtie  R_2$.
\end{itemize}

\begin{remark}
	Как уже говорилось, все операции в рамках алгебры замкнуты по определению. Это означает, что их можно комбинировать произвольным образом (при сохранении условий на возможность исполнения операции). Например: $\pi_A(R_1      \bowtie  \pi_B(R_2)) \bowtie  R_3$.
\end{remark}

\paragraph{Операции}
В текущем контексте полезно уточнить, что именно понимается под операцией над отношениями в рамках реляционной алгебры. А именно, для того, чтобы определить операцию, необходимо определить следующее:

\begin{itemize}
	\item Правило построения заголовка по заданным отношениям;
	\item Правило построения тела по заданным отношениям;
	\item Условия, при которых операция выполнима, то есть ограничения на отношения, к которым она применяется.
\end{itemize}
