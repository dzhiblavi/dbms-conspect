\subsection{Исчисление кортежей и его реляционная полнота}

\subsubsection{Реляционное исчисление}

\begin{definition}
	\textit{Реляционное исчисление} -- декларативный язык для работы с отношениями.
\end{definition}

В отличие от реляционной алгебры, в реляционном исчислении описывается не то, как построить запрос,
а свойства результата, который ожидаем получить.

Выделяют два вида реляционного исчисления: \textbf{исчисление кортежей} и \textbf{исчисление доменов}.

\paragraph{Части запроса}

\begin{itemize}
	\item \textbf{Определение переменных},
	\item \textbf{Определение свойств результата}.
\end{itemize}

\paragraph{Синтаксис запроса}

\begin{lstlisting}[language=SQL]
    <variables definition>
    select <attributes list>
    from <variables>
    where <condition>
\end{lstlisting}

\paragraph{Пример}

Рассмотрим запрос, для получения идентификаторов студентов, обучающихся в группе
\texttt{M34371}.

Запрос в реляционной алгебре:

\[
	\pi_{SId}(\sigma_{Name = M34371} (Students \bowtie Groups))
\]

Запрос в реляционном исчислении:

\begin{lstlisting}[language=SQL, mathescape=true]
    select S.SId from S
    where $\exists$ G (S.GId = G.GId $\wedge$ G.Name = 'M34371')
\end{lstlisting}

\subsubsection{Исчисление кортежей}

В качестве переменных выступают кортежи. Тип переменной определяется именем и типами атрибутов, а
также набором значений. Заметим, что это есть отношение.

\paragraph{Синтаксис}

\begin{lstlisting}
    <Variable> :: <Relation>
\end{lstlisting}

\paragraph{Операции с отношениями}

\begin{itemize}
	\item \textbf{Огранричение}
	      \begin{lstlisting}[language=SQL]
    <Relation> where <Condition>
        \end{lstlisting}
	\item \textbf{Объединение}
	      \begin{lstlisting}[language=SQL]
    <Relation 1>, <Relation 2>
        \end{lstlisting}
\end{itemize}

\paragraph{Примеры}

Рассмотренное выше отношение \texttt{G4} можно выразить иначе:

\begin{lstlisting}[language=SQL, mathescape=true]
    G4 :: Groups where Name = 'M34351',
          Groups where Name = 'M34371',
          Groups where Name = 'M34391'
\end{lstlisting}

\paragraph{Условия}

Условия во многом аналогичны условиям из реляционной алгебры. Дополнительно вводятся кванторы.

\begin{itemize}
	\item \textbf{Простые условия}.
	      \begin{itemize}
		      \item Сравнение атрибутов с константами
		            \begin{lstlisting}[language=SQL, mathescape=true]
    S.Name = 'John'
    S.Id $<$ 5
                \end{lstlisting}
		      \item Сравнение атрибутов между собой
		            \begin{lstlisting}[language=SQL, mathescape=true]
    S.Id $\geq$ G.Id
                \end{lstlisting}
		      \item Сравнение с применением формул
		            \begin{lstlisting}[language=SQL, mathescape=true]
    length(S.FirstName) = length(S.LastName) + 3
                \end{lstlisting}
	      \end{itemize}
	\item \textbf{Составные условия}. Вводятся логические связки: $\vee$, $\wedge$,
	      $\neg$.
	      \begin{lstlisting}[language=SQL, mathescape=true]
    G where Name = 'M34371' $\vee$ Name = 'M34391'
    S where FirstName = 'John' $\wedge$ LastName <> 'Smith'
        \end{lstlisting}
	\item \textbf{Условия с кванторами}. Вводятся кванторы всеобщности $\forall$ и существования
	      $\exists$. Синтаксис: \texttt{<quantifier> <variable> (<condition>)}.
	      \begin{lstlisting}[language=SQL, mathescape=true]
    G where $\exists$ S (S.FirstName = 'John' $\wedge$ S.GId = G.GId)
    G where $\forall$ S (S.FirstName = 'John' $\vee$ S.GId = G.GId)
        \end{lstlisting}
\end{itemize}

\paragraph{Примеры}

Рассмотрим примеры введения переменных, запроса, который требует в реляционной алгебре применения
деления, и работы с несколькими отношениями.

\begin{lstlisting}[language=SQL, mathescape=true]
    -- Variables declaration
    S :: Students; G :: Groups; C :: Courses; P :: Point;
    G4 :: Groups where Name = 'M34351' $\vee$
        Name = 'M34371' $\vee$ Name = 'M34391'

    -- Fully marked groups
    select G.GId from G where $\forall$ S ($\forall$ C($\exists$ P
        (S.Sid = P.Sid $\wedge$ C.CId = P.CId $\wedge$ P.Points $\geq$ 60)))

    -- Multiple relations
    select S.FirstName, S.LastName, G.Name
    from S, G
    where S.GId = G.GId
\end{lstlisting}

\subsubsection{Связи реляционной алгебры и исчисления кортежей}

\paragraph{Алгебра через исчисление}

Выразим реляционную алгебру через исчисление кортежей.

\begin{itemize}
	\item \textbf{Проекция}.
	      \begin{itemize}
		      \item $\pi_{A_1, \ldots, A_n}(R)$
		      \item \texttt{select A1, ..., An from R}
	      \end{itemize}
	\item \textbf{Фильтрация}.
	      \begin{itemize}
		      \item $\sigma_{\theta}(R)$
		      \item \texttt{from R where $\theta$}
	      \end{itemize}
	\item \textbf{Создание нового столбца}.
	      \begin{itemize}
		      \item $\epsilon_{A=expr}(R)$
		      \item \texttt{select R.*, expr as A from R}
	      \end{itemize}
	\item \textbf{Объединение}.
	      \begin{itemize}
		      \item $R_1 \cup R_2$
		      \item \texttt{R :: R1, R2}
	      \end{itemize}
	\item \textbf{Разность}.
	      \begin{itemize}
		      \item $R_1 \setminus R_2$
		      \item \texttt{R :: R1 where $\neg \exists$ R2 (R1 = R2)}
	      \end{itemize}
	\item \textbf{Декартово произведение}.
	      \begin{itemize}
		      \item $R_1 \times R_2$
		      \item \texttt{select R1.*, R2.* from R1, R2}
	      \end{itemize}
	\item \textbf{Объединение}.
	      \begin{itemize}
		      \item $R_1 \bowtie R_2$
		      \item \texttt{select R1.*, R2.* from R1, R2}
		            \linebreak \texttt{where R1.<common\_attrs> = R2.<common\_attrs>}
	      \end{itemize}
\end{itemize}

Указанные выше операции образуют базис реляционной алгебры. Следовательно, выразительная мощность
реляционной алгебры \textit{не больше} выразительной мощности реляционного исчисления, и
алгебру можно выразить через исчисление.

\paragraph{Исчисление через алгебру}

Выражения с кванторами можно привести к предваренной нормальной форме (сначала идут только
кванторы, затем -- условие). После этого можно преобразовать выражение в алгебру:

\begin{itemize}
	\item Построение выражения для каждой переменной,
	\item Построение декартового произведения по всем затронутым отношениям,
	\item Фильтрация по условию,
	\item Применение кванторов:
	      \begin{itemize}
		      \item Квантор существования: проекция, исключая атрибуты, порожденные переменной. Если существует хотя бы
		            одно значение кортежной переменной, удовлетворяющее условию, то проекция будет непустой. Верно
		            обратное.
		      \item Квантор всеобщности: как было замечено, логике этого квантора соответствует деление. Таким образом,
		            достаточно делить на все столбцы кортежной переменной.
	      \end{itemize}
\end{itemize}

\paragraph{Пример}

Запрос получает идентификаторы групп, в которых есть хотя бы один студент, аттестованный по всем
дисциплинам.

\begin{lstlisting}[language=SQL, mathescape=true]
    select G.GId where $\exists$ S($\forall$ C($\exists$ P
        (G.GId = S.GId $\wedge$ S.SId = P.SId $\wedge$
         C.CId = P.CId $\wedge$ P.Points $\geq$ 60)))
\end{lstlisting}

Преобразуем его в реляционную алгебру.

\[
	T_0 = \sigma_{P.Points \geq 60}((G \bowtie S \times C) \bowtie P)
\]

Самый внутренний квантор ($\exists P$) -- проекция на все внешние атрибуты (находящиеся
левее в выражении):

\[
	T_1 = \pi_{G_*, S_*, C_*}(T_0)
\]

Далее следует квантор всеобщности ($\forall C$) -- деление на переменную:

\[
	T_2 = T_1 \div C
\]

Последний квантор существования ($\exists S$) -- проекция на единственный оставшийся
внешний атрибут:

\[
	T_3 = \pi_{G_*}(T_2)
\]

Получили выражение в терминах реляционной алгебры:

\[
	T = \pi_{G_*}(\pi_{G_*, S_*, C_*} (\sigma_{P.Points \geq 60}((G \bowtie S \times C) \bowtie P))
	\div C)
\]

\begin{proposition}
	Выразительная мощность исчисления кортежей равна выразительной мощности реляционной алгебры.
\end{proposition}

\begin{definition}
	\textit{Реляционно-полные языки} -- языки, выразительная мощность которых не меньше
	выразительной мощности реляционной алгебры.
\end{definition}

Следовательно, исчисление кортежей -- реляционно-полный язык.
