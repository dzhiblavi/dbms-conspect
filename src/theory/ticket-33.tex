\subsection{Транзакции. Параллельное исполнение. Уровни изоляции.}

\paragraph{Уровни изоляции транзакций}

Мы рассматриваем следующие уровни изоляции транзакций. Все, кроме ``Слепок'', определены в
стандарте SQL.

\begin{itemize}
	\item \textbf{Упорядочиваемый (serializable)},
	\item \textbf{``Слепок'' (snapshot)},
	\item \textbf{Повторяемое чтение (repeatable read)},
	\item \textbf{Чтение зафиксированных (read committed)},
	\item \textbf{Чтение незафиксированных (read uncommitted)}.
\end{itemize}

\subsubsection{Упорядочиваемый}

Дает наиболее сильные гарантии с самой низкой скоростью исполнения. Детали реализации были
рассмотрены в предыдущем билете.

\subsubsection{``Слепок''}

Каждая транзакция работает со своим ``слепком'' БД. Вносит изменения в режиме copy-on-write. При
реинтеграции изменений они фиксируются, при отсутствии конфликтов изменений.

Является аналогом упорядочиваемого уровня изоляции с меньшими гарантиями, используется в
базах-``версионниках'', в которых синхронизация основана на версиях вместо блокировок.

\paragraph{Аномалия ``косая запись''}
На данном уровне изоляции возможна аномалия ``косая запись''. Она возникает при одновременном
обновлении разных записей, которые вместе должны гарантировать некоторый инвариант.

\textbf{Пример}. Положим инвариант $\texttt{t\_1} + \texttt{t\_2} \geq 0$.

\begin{itemize}
	\item Транзакция 1
	      \begin{lstlisting}[language=SQL]
            if t_1 + t_2 >= DELTA begin
                t_1 = t_1 - DELTA
            end if
        \end{lstlisting}
	\item Транзакция 2
	      \begin{lstlisting}[language=SQL]
            if t_1 + t_2 >= DELTA begin
                t_2 = t_2 - DELTA
            end if
    \end{lstlisting}
\end{itemize}

Реинтеграция изменений пройдет успешно, поскольку записи идут в разные переменные. Однако, при
параллельном исполнении инвариант может быть нарушен.

\subsubsection{Повторяемое чтение}

Уровень изоляции гарантирует, что при повторном чтении значения не будут меняться. Исключение --
запись, произведенная самой транзакцией. Реализуется путем взятия блокировок записей или страниц на
чтение.

\paragraph{Аномалия ``фантомная запись''}

При повторном чтении могут появиться новые записи. Возможно при параллельном исполнении другой
транзакции.

\subsubsection{Чтение зафиксированных}

Уровень изоляции гарантирует, что читаемые значения зафиксированы другими транзакциями. Реализуется
путем взятия \textit{частичных} блокировок записей или страниц на чтение.

\paragraph{Аномалия ``неповторяемое чтение''}

При повторном чтении могут значение записи может измениться. Возможно при параллельном исполнении
другой транзакции.

\subsubsection{Чтение незафиксированных}

На уровне изоляции не используются блокировки, что обеспечивает наивысшую скорость. По стандарту
SQL разрешено только чтение. Используется для сбора статистики.

\paragraph{Аномалия ``грязное чтение''}

Может быть прочитано некорректное значение.
