\subsection{Секционирование}

\begin{definition}
	\textit{Секционирование} -- разбиение таблицы на фрагменты, хранящиеся в разных местах (в
	случае разных компьютеров называется \textit{шардинг}). Используется для увеличения скорости
	чтения за счет параллельного обращения.
\end{definition}

Различают два вида секционирования: \textbf{вертикальное} и \textbf{горизонтальное}.

\subsubsection{Вертикальное секционирование}

Таблица разбивается по столбцам. Возможно при корректности соединения (5 НФ). Реализуется
посредством проекции и соединения.

\paragraph{Преимущества}

\begin{itemize}
	\item Отделение данных, к которым часто обращаются, от тех, к которым обращаются редко.
	\item Защита информации.
	\item Поддерживается во многих СУБД для \texttt{CLOB} и \texttt{BLOB}.
\end{itemize}

\paragraph{Недостатки}

\begin{itemize}
	\item Нет специальной поддержки в СУБД. Считается, что проекции и соединения для указанных целей
	      достаточно.
	\item Зависимость от представления (соединенных данных). Некоторые СУБД накладывают ограничения на
	      представления, например, запрещают создавать внешние ключи на них.
	\item Необходимость обновляемых представлений также не гарантирована.
\end{itemize}

\paragraph{Пример}

Рассмотрим исходную таблицу.

\begin{lstlisting}
    Students(SId, GId, FirstName, LastName, PassSeries,
        PassNo, PassIssued, Photo)
\end{lstlisting}

Разобьем ее на секции.

\begin{lstlisting}
    StudentData(SId, GId, FirstName, LastName)
    StudentPasses(SId, PassSeries, PassNo, PassIssued)
    StudentPhotos(SId, Photo)
\end{lstlisting}

Обращение к фото (\texttt{StudentPhotos}) происходит значительно реже, чем к основным данным
студента (\texttt{StudentData}). Также таблица с персональными данными (\texttt{StudentPasses})
требует повышенных прав доступа. Таким образом, были использованы все преимущества вертикального
секционирования.

Создадим также представление для работы с исходной таблицей.

\begin{lstlisting}[language=SQL]
    create view Students as StudentData
        natural join StudentPasses
        natural join StudentPhotos;
\end{lstlisting}

\subsubsection{Горизонтальное секционирование}

Таблица разделяется по строкам. Корректно, когда каждая строка попадает ровно в одну секцию.
Реализуется посредством фильтрации и объединения.

\paragraph{Преимущества}

\begin{itemize}
	\item Отделение данных, к которым часто обращаются, от тех, к которым обращаются редко. Например, чаще
	      всего старые данные нужны реже новых.
	\item При уменьшении размера секции уменьшается размер индекса.
	\item Требуется встроенная поддержка.
	\item Прозрачно для пользователя.
\end{itemize}

\paragraph{Недостатки}

\begin{itemize}
	\item В некоторых случаях может приводить к замедлению работы.
\end{itemize}

\paragraph{Пример}

Рассмотрим исходную таблицу.

\begin{lstlisting}
    Points(SId, CId, Points, Date)
\end{lstlisting}

Введем секционирование по Date:

\begin{itemize}
	\item $\texttt{Points}_{2021-1}$ -- оценки за весенний семестр 2021,
	\item $\texttt{Points}_{2021-2}$ -- оценки за осенний семестр 2021,
	\item $\texttt{Points}_{2020-1}$ -- оценки за весенний семестр 2020,
	\item \ldots
\end{itemize}

\paragraph{Методы секционирования}

\begin{itemize}
	\item \textbf{Простые}.
	      \begin{itemize}
		      \item По диапазонам значений,
		      \item По значениям,
		      \item По хешу.
	      \end{itemize}
	\item \textbf{По выражению}. Поддерживаются реже.
	\item \textbf{Составные}.
	      \begin{itemize}
		      \item По диапазонам и хешу,
		      \item \ldots
	      \end{itemize}
\end{itemize}

\paragraph{Пример} Секционирование по диапазонам.

\begin{lstlisting}[language=SQL]
    create table Points(...)
    partition by range(Date) (
        partition pHist values less than '2021-02-01',
        partition p2021s1 values less than '2021-07-01',
        partition p2021s2 values less than '2022-02-01',
        partition pFuture values less than maxvalue
    );
\end{lstlisting}

\texttt{maxvalue} -- максимальное теоретическое значение.

\paragraph{Пример} Секционирование по значениям.

Чаще используется для перечислений.

\begin{lstlisting}[language=SQL]
    create table Points(...)
    partition by list(Term) (
        partition pHist values in ('t2020-1', 't2020-2', ...),
        partition p2021s1 values in ('t2021-1'),
        partition p2021s2 values in ('t2021-2'),
        partition pFuture values in ('t2022-1')
    );
\end{lstlisting}

При таком подходе секция может быть не определена при записи.

\begin{lstlisting}[language=SQL]
    insert into Points (Term) values ('t2001-1');
\end{lstlisting}

При чтении из несуществующей секции будет получен пустой результат.

\paragraph{Пример} Секционирование по хешу.

Хешируется по набору столбцов. Работает эффективно при хорошей и быстрой хеш-функции.

\begin{lstlisting}[language=SQL]
    create table Points(...)
    partition by hash(Term)
    partitions 4;
\end{lstlisting}

\paragraph{Пример} Секционирование по выражению.

Зависит от определенной в БД функции.

\begin{lstlisting}[language=SQL]
    create table Points(...)
    partition by year(Date) (
        partition pHist values less than 2021,
        partition pCurrent values less than 2022,
        partition pFuture values less than maxvalue
    );
\end{lstlisting}

\paragraph{Пример} Составное секционирование.

Секции разбиваются на подсекции, возможно, по разным атрибутам.

\begin{lstlisting}[language=SQL]
    create table Points(...)
    partition by year(Date)
    subpartition by hash(Term) (
        partition History values less than 2021 (
            subpartition History1, subpartition History2
        ),
        partition Current values less than 2022 (
            subpartition Current1, subpartition Current2
        )
    );
\end{lstlisting}

\paragraph{Управление секциями}

\begin{remark}
	Данные команды не входят в стандарт SQL, поэтому синтаксис может отличаться в зависимости от СУБД.
\end{remark}

\textbf{Удаление секции}

\begin{lstlisting}[language=SQL]
    alter table <table> drop partition <section>;
\end{lstlisting}

\textbf{Разбиение секции}

\begin{lstlisting}[language=SQL]
    alter table <table> reorganize <section> into (...);
\end{lstlisting}

\textbf{Перехеширование}

\begin{lstlisting}[language=SQL]
    -- Add count of partitions
    alter table <table> add partition <count>;
    -- Delete count of partitions
    alter table <table> coalesce partition <count>;
\end{lstlisting}

\begin{proposition}
	Оптимизатор владеет информацией о секциях. В частности, где какие данные находятся.
\end{proposition}

\paragraph{Пример}

\begin{lstlisting}[language=SQL]
    select * from Points where Date = '2021-12-06
\end{lstlisting}

При таком запросе выборка будет производиться только из секции 2021 года.

\paragraph{Секционирование и индексы}

Можно определить следующие индексы:

\begin{itemize}
	\item \textbf{Локальный} -- один на секцию. Ускорение при выборе секций.
	\item \textbf{Глобальный} -- один на таблицу. Также ускорение при выборе секций.
	\item \textbf{Секционированный} -- разбит на секции. Обеспечивает согласованное
	      секционирование (могут помочь при склеивании секций).
\end{itemize}

\paragraph{Секционирование и ключи}

Лучше всего, если множество столбцов, по которым идет секционирование, образует подключ. Еще лучше
-- подключ всех ключей (например, если внешние ключи на таблицу ссылаются на разные ее ключи).
