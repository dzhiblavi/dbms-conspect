\subsection{Исчисление доменов и его реляционная полнота}

\subsubsection{Исчисление доменов}

В качестве переменных выступают домены (типы). Иными словами, работа ведется с атомарными
переменными в отличие от исчисления кортежей, где работа ведется с отношениями.

\paragraph{Синтаксис объявления переменных}

Указывается имя и \textit{домен (тип)}.

\begin{lstlisting}
    <Variable> :: <Type>
\end{lstlisting}

\paragraph{Пример}

Можно использовать типы из SQL.

\begin{lstlisting}[language=SQL]
    SId :: Int
    FirstName :: Varchar(100)
\end{lstlisting}

\paragraph{Условие принадлежности}

Для связи доменов с отношениями вводится предикат проверки принадлежности значения заданному
кортежу.

\paragraph{Синтаксис}

Отсутствие какого-либо атрибута отношения означает, что его конкретное значение нас не интересует.

\begin{lstlisting}[language=SQL]
    <Relation> {
        <Attribute1> = <Value1>,
        <Attribute2> = <Value2>,
        ...
    }
\end{lstlisting}

\paragraph{Пример}

В последнем примере \texttt{SId = SId} означает, что аттрибут \texttt{SId} (слева) равен
значению переменной с именем \texttt{SId}. Не отличается по смыслу от предыдущего.

\begin{lstlisting}[language=SQL]
    S{FirstName = 'John', LastName = 'Smith'}
    S{SId = Id}
    S{SId = SId}
\end{lstlisting}

\paragraph{Примеры}

\begin{itemize}
	\item Идентификаторы всех студентов:
	      \begin{lstlisting}[language=SQL]
    SId where S{SId = SId}
        \end{lstlisting}
	\item Идентификаторы всех студентов определенной группы:
	      \begin{lstlisting}[language=SQL]
    SId where S{SId = SId, GId = 'M34371'}
        \end{lstlisting}
	\item Идентификаторы всех студентов двух определенных групп:
	      \begin{lstlisting}[language=SQL, mathescape=true]
    SId where S{SId = SId, GId = 'M34371'} $\vee$
              S{SId = SId, GId = 'M34391'}
        \end{lstlisting}
	\item Идентификаторы всех студентов, не сдавших курс с определенным идентификатором.
	      \begin{lstlisting}[language=SQL, mathescape=true]
    SId where $\neg \exists$ Points (Points $\geq$ 60 $\wedge$
        P {SId = SId, Points = Points, CId = 10})
        \end{lstlisting}
\end{itemize}

\subsubsection{Связи реляционной алгебры и исчисления доменов}

\paragraph{Алгебра через исчисление}

Выразим реляционную алгебру через исчисление доменов.

\begin{itemize}
	\item \textbf{Проекция}.
	      \begin{itemize}
		      \item $\pi_{A_1, \ldots, A_n}(R)$
		      \item \texttt{A1, ..., An from R where R\{A1 = A1, ..., An = An\}}
	      \end{itemize}
	\item \textbf{Фильтрация}.
	      \begin{itemize}
		      \item $\sigma_{\theta}(R)$
		      \item \texttt{A1, ..., An from R where R\{A1 = A1, ..., An = An\} $\wedge$ $\theta$}
	      \end{itemize}
	\item \textbf{Создание нового столбца}.
	      \begin{itemize}
		      \item $\epsilon_{A=expr}(R)$
		      \item \texttt{expr as A from R where R\{A1 = A1, ..., An = An\}}
	      \end{itemize}
	\item \textbf{Объединение}.
	      \begin{itemize}
		      \item $R_1 \cup R_2$
		      \item \texttt{A1, ..., An where R1\{..., Ai = Ai, ...\} $\vee$ R2\{..., Ai = Ai, ...\}}
	      \end{itemize}
	\item \textbf{Разность}.
	      \begin{itemize}
		      \item $R_1 \setminus R_2$
		      \item \texttt{A1, ..., An where R1\{..., Ai = Ai, ...\} $\vee$ $\neg$
			            R2\{..., Ai = Ai, ...\}}
	      \end{itemize}
	\item \textbf{Декартово произведение}.
	      \begin{itemize}
		      \item $R_1 \times R_2$
		      \item \texttt{A1, ..., An, B1, ..., Bm where R1\{..., Ai = Ai, ...\} $\wedge$}
		            \linebreak \texttt{R2\{..., Bj = Bj, ...\}}
	      \end{itemize}
	\item \textbf{Объединение}.
	      \begin{itemize}
		      \item $R_1 \bowtie R_2$
		      \item \texttt{A1, ..., An, B1, ..., Bm, C1, ..., Cl}
		            \linebreak \texttt{where R1\{..., Ai = Ai, ..., Bj = Bj, ...\} $\wedge$}
		            \linebreak \texttt{R2\{..., Bj = Bj, ..., Ck = Ck, ...\}}
	      \end{itemize}
\end{itemize}

\begin{proposition}
	Выразительная мощность исчисления доменов не меньше выразительной мощности реляционной алгебры, то
	есть исчисление доменов -- реляционно-полный язык.
\end{proposition}
