\subsection{Datalog и рекурсия}

\subsubsection{Datalog}

\textit{Datalog} -- язык запросов к БД. Разработан в 1978 году, имеет много схожего с языком
Prolog. Повлиял на развитие SQL. Реализует исчисление доменов.

Программа на языке состоит из набора определений отношений. Результатом считается тело одного из
отношений.

\paragraph{Конструкции языка}

\begin{itemize}
	\item \textbf{Реляционные атомы}. Реализуют предикат вхождения из исчисления доменов.
	      Используется позиционная запись аргументов.
	      \begin{lstlisting}[mathescape=true]
    -- Values belong to relation R
    R(x1, x2, ..., xn)
    -- Values do not belong to relation R
    $\neg$ R(x1, x2, ..., xn)
        \end{lstlisting}
	\item \textbf{Арифметические атомы}. Сравнение двух выражений.
\end{itemize}

\paragraph{Синтаксис}

Программа состоит из списка строк:

\begin{lstlisting}
    <Relation>(<Attr 1>, <Attr 2>, ..., <Attr n>) :- <Target>
\end{lstlisting}

Цель (target) -- последовательность атомов. В отношение (Relation) входят кортежи, удовлетворяющие
хотя бы одной цели (всем атомам хотя бы одной цели).

Иначе говоря, запросы пишутся в дизъюнктивной нормальной форме.

\paragraph{Примеры}

\begin{itemize}
	\item Идентификаторы и фамилии всех студентов с заданным именем:
	      \begin{lstlisting}
    Johns(Id, LastName) :-
        Students(Id, FirstName, LastName),
        FirstName = 'John'.
        \end{lstlisting}
	\item Имена всех студентов и преподавателей:
	      \begin{lstlisting}
    Names(Name) :- Students(_, Name, _).
    Names(Name) :- Lecturers(_, Name, _).
        \end{lstlisting}
\end{itemize}

\paragraph{Бесконечные отношения}

Запросы могут породить бесконечные отношения. Например:

\begin{lstlisting}[mathescape=true]
    NotStudent(Id, Name) :- $\neg$ Students(Id, Name, _).
\end{lstlisting}

Здесь нет ограничения на фамилию студента -- ни по типу, ни по значению -- поэтому это отношение
бесконечно и содержит студентов с почти произвольными фамилиями.

\begin{proposition}
	Для запрета бесконечных отношения достаточно принудить каждую переменную входить как минимум в один
	неотрицательный реляционный атом.
\end{proposition}

\paragraph{Реляционная алгебра через Datalog}

Выразим реляционную алгебру через исчисление доменов.

\begin{itemize}
	\item \textbf{Проекция}.
	      \begin{itemize}
		      \item $\pi_{A_1, \ldots, A_n}(R)$
		      \item \texttt{Q(A1, ..., An) :- R(A1, ..., An, \_, ..., \_).}
	      \end{itemize}
	\item \textbf{Фильтрация}.
	      \begin{itemize}
		      \item $\sigma_{\theta}(R)$
		      \item \texttt{Q(A1, ..., An) :- R(A1, ..., An, \_, ..., \_), $\theta$.}
	      \end{itemize}
	\item \textbf{Объединение}.
	      \begin{itemize}
		      \item $R_1 \cup R_2$
		      \item \texttt{Q(A1, ..., An) :- R1(A1, ..., An).}
		            \linebreak \texttt{Q(A1, ..., An) :- R2(A1, ..., An).}
	      \end{itemize}
	\item \textbf{Разность}.
	      \begin{itemize}
		      \item $R_1 \setminus R_2$
		      \item \texttt{Q(A1, ..., An) :- R1(A1, ..., An), $\neg$ R2(A1, ..., An).}
		      \item \textbf{Замечание}. Запись корректна, поскольку атрибуты \texttt{A1, ..., An} входят
		            в неотрицательный реляционный атом R1.
	      \end{itemize}
	\item \textbf{Декартово произведение}.
	      \begin{itemize}
		      \item $R_1 \times R_2$
		      \item \texttt{Q(A1, ..., An, B1, ..., Bm) :- R1(A1, ..., An), R2(B1, ..., Bm).}
	      \end{itemize}
	\item \textbf{Объединение}.
	      \begin{itemize}
		      \item $R_1 \bowtie R_2$
		      \item \texttt{Q(A1, ..., An, B1, ..., Bm, C1, ..., Cl) :-}
		            \linebreak \texttt{R1(A1, ..., An, B1, ..., Bm), R2(B1, ..., Bm, C1, ..., Cl).}
	      \end{itemize}
\end{itemize}

\begin{proposition}
	Datalog -- реляционно-полный язык.
\end{proposition}

\begin{remark}
	В Datalog нет явных кванторов. Есть только неявный квантор существования -- для свободных
	переменных справа в выражении.
\end{remark}

\paragraph{Пример}

Рассмотрим таблицу:

\begin{lstlisting}
    Person(Id, Name, MotherId, FatherId)
\end{lstlisting}

Получить для каждого человека имена обоих родителей:

\begin{lstlisting}
    Parents(N, FN, MN) :- Person(_, N, FId, MId),
        Person(FId, FN, _, _), Person(MId, MN, _, _).
\end{lstlisting}

Получить для каждого человека имя его родителя:

\begin{lstlisting}
    Parents(N, FN) :- Person(_, N, FId, _), Person(FId, FN, _, _).
    Parents(N, MN) :- Person(_, N, MId, _), Person(MId, MN, _, _).
\end{lstlisting}

\subsubsection{Рекурсивные запросы}

Datalog позволяет строить рекурсивные запросы, например, для транзитивного замыкания. Рассмотрим
для примера таблицу:

\begin{lstlisting}
    Parent(Id, ParentId)
\end{lstlisting}

И запрос для получения всех предков (представляющий собой транзитивное замыкание):

\begin{lstlisting}
    Ancestor(Id, PId) :- Parent(Id, PId).
    Ancestor(Id, GId) :- Parent(Id, PId), Ancestor(PId, GId).
\end{lstlisting}

Такому запросу отвечает несколько множеств кортежей. Формально, помимо ожидаемого транзитивного
замыкания, декартово произведение всех возможных идентификаторов будет удовлетворять запросу. При
этом, такое множество кортежей будет \textit{неподвижной точкой} -- множеством, которое нельзя дополнить
по правилам вывода. Но такой результат не будет иметь практического смысла

Во избежание неопределенности, в Datalog результатом рекурсивного запроса является
\textit{минимальная неподвижная точка} -- минимальное по включению множество, такое что левая и правая части
совпадают.

На практике рекурсивное правило применяется, пока не будет достигнута неподвижная точка. Очевидно,
она будет минимальной по включению.

\paragraph{Поиск минимальной неподвижной точки}

\begin{itemize}
	\item Инициализация нерекурсивными данным,
	\item Пока есть следствия, данные  пополняются ими.
\end{itemize}

\begin{remark}
	Отрицания в рекурсивных правилах могут приводить к парадоксам (например, на шаге
	$n$ какие-то данные должны быть добавлены, на $n + 1$ -- нет).
	Поэтому в рекурсивных запросах явно запрещены циклы, содержащие отрицания.
\end{remark}

\begin{remark}
	Как было замечено ранее, реляционная алгебра не позволяет построить рекурсивные запросы.
	Следовательно, выразительная мощность Datalog больше, чем у реляционной алгебры.
\end{remark}

\subsubsection{Связь реляционного исчисления и SQL}

\begin{itemize}
	\item \textbf{Заголовок запроса}. Определяет, какие данные будут получены. В реляционной
	      алгебре соответствует самой внешней проекции. В реляционном исчислении -- заголовку запроса.
	\item \textbf{Раздел \texttt{from}}. Определяет набор отношений, с которыми производится
	      работа. В реляционной алгебре здесь указываются соединения. В реляционном исчислении -- переменные
	      отношений.
	\item \textbf{Раздел \texttt{where}}. Определяет набор отношений, с которыми производится
	      работа. В реляционной алгебре соответствует самой внешней фильтрации. В реляционном исчислении --
	      условию на кортежи.
\end{itemize}

\begin{remark}
	В языке SQL на практике большинство кванторов применяется для соединений. Однако, существует также
	явный квантор существования. Как было показано в Datalog, квантор всеобщности избыточен
	($\forall v(p(v)) \equiv \neg \exists v(\neg p(v))$)
\end{remark}

\subsubsection{Подзапросы}

Квантор существования предоставлен в SQL через подзапросы существования:

\begin{itemize}
	\item Существования (\texttt{exists}),
	\item Вхождения (\texttt{in}),
	\item Условные (\texttt{any, all}),
	\item Скалярные.
\end{itemize}

\paragraph{Подзапрос существования}

Если подзапросу удовлетворяет хотя бы один кортеж, то оператор возвращает \texttt{true},
иначе \texttt{false}.

\begin{lstlisting}[language=SQL]
    [not] exists (select ...)
\end{lstlisting}

\paragraph{Пример}

Группы, в которых нет студентов:

\begin{lstlisting}[language=SQL]
    select G.Name where
    not exists (select * from S where S.Gid = G.Gid)
\end{lstlisting}

\paragraph{Подзапрос вхождения}

Если результаты всех выражений входят в подзапрос, то оператор возвращает \texttt{true},
иначе \texttt{false}.

\begin{lstlisting}[language=SQL]
    (<list of expressions>) [not] in (select ...)
\end{lstlisting}

\begin{remark}
	Многие СУБД поддерживают подзапросы только с одним столбцом.
\end{remark}

\paragraph{Пример}

Оценки по предметам 4 курса:

\begin{lstlisting}[language=SQL]
    select * from P where P.CId in
        (select CId from C where Year = 4)
\end{lstlisting}

\paragraph{Условный подзапрос}

\texttt{any} возвращает \texttt{true}, если для одного из значений выполняется
условие. Аналогично, \texttt{all} возвращает \texttt{true}, если для всех
значений выполняется условие.

\begin{lstlisting}[language=SQL]
    <expression> <condition> { any | all } (select ...)
\end{lstlisting}

\paragraph{Пример}

Лучшие студенты по предметам:

\begin{lstlisting}[language=SQL]
    select SId, CId from P as PE where PE.Points >=
        all (select Points from P where P.CId = PE.CId)
\end{lstlisting}

\paragraph{Скалярные подзапросы}

Запросы, возвращающие не более одного значения.

\begin{lstlisting}[language=SQL]
    (select ...)
\end{lstlisting}

\begin{remark}
	Результатом скалярного подзапроса, вернувшего пустой результат, является \texttt{null}.
\end{remark}

\paragraph{Пример}

Студенты и название группы:

\begin{lstlisting}[language=SQL]
    select
        SId,
        (select Name from G where G.GId = S.GId)
    from S
\end{lstlisting}

\paragraph{Кореллированные подзапросы}

\begin{definition}
	Подзапрос называется \textit{кореллированным (некореллированным)}, если в нем есть (отсутствуют) свободные переменные.
\end{definition}

\paragraph{Примеры}

Лучшие студенты по предметам (кореллированный подзапрос, \texttt{PE} -- свободная
переменная):

\begin{lstlisting}[language=SQL, escapechar=\%]
    select SId, CId from P as %\underline{PE}% where PE.Points >=
        all (select Points from P where P.CId = %\underline{PE}%.CId)
\end{lstlisting}

Оценки по предметам четвертого курса (некореллированный подзапрос):

\begin{lstlisting}[language=SQL]
    select * from P where P.CId in
        (select CId from C where Year = 4)
\end{lstlisting}

\begin{remark}
	Преимущество некореллированных подзапросов в том, что их можно посчитать один раз и
	переиспользовать результат в дальнейшем. Поэтому на практике они предпочтительнее кореллированных.
\end{remark}

\subsubsection{Рекурсия}

Поддержка рекурсии была добавлена в SQL3. Однако, многие СУБД не поддерживают ее.

\paragraph{Синтаксис}

Как и в Datalog, отрицание должно быть стратифицировано -- оно запрещено в циклах рекурсии.

\begin{lstlisting}[language=SQL]
    with recursive <relation>(<columns>) as ...
\end{lstlisting}

\paragraph{Пример}

Транзитивное замыкание предков:

\begin{lstlisting}[language=SQL]
    with recursive Ancestor(Id, AId) as
        select Id, PId from Parent
        union
        select P.Id, AId from Ancestor a
            inner join Parent p on a.Id = p.PId
    select * from Ancestor
\end{lstlisting}
