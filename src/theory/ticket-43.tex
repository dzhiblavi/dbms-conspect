\subsubsection{Операции с null в SQL}

\paragraph{Предикаты}

\begin{itemize}
	\item \textbf{DML}. В результате \texttt{where} и \texttt{having} остается только
	      \texttt{true}. Все другие значения преобразовываются в \texttt{false}, включая
	      \texttt{unknown}.
	\item \textbf{DDL}. В \texttt{check} не подходят только \texttt{false}. Все другие
	      значения
	      преобразовываются в \texttt{true}, включая \texttt{unknown}. То есть ограничение,
	      вернувшее \texttt{unknown} считается удовлетворяющим.
\end{itemize}

\paragraph{Различимость}

Сравнение двух \texttt{null} возвращает \texttt{unknown}. При использовании его в
\texttt{where} результат преобразуется в \texttt{false}. Следовательно, два
\texttt{null} не равны. Кроме того, они не различимы.

Неразличимость не эквивалентна неравенству. Например, при использовании \texttt{distinct} из
двух одинаковых кортежей с отсутствующими полями в результате будет один. Аналогично,
\texttt{group by} породит только одну корзину.

\paragraph{Столбцы}

По умолчанию, столбцы допускают \texttt{null} в качестве значения:

\begin{lstlisting}[language=SQL]
    birthday date
\end{lstlisting}

Для запрета отсутствующих значений это требуется явно указать:

\begin{lstlisting}[language=SQL]
    birthday date not null
\end{lstlisting}

\paragraph{Проверки значений}

Для надежной проверки, чему равно значение из тернарной логики, используется следующий оператор:

\begin{lstlisting}[language=SQL]
    <value> is [not] {null | true | false | unknown}
\end{lstlisting}

\texttt{is null} -- способ проверить, равно ли что-либо \texttt{null}.
\texttt{is unknown}
применяется для булевских значений. Однако, для булевских значений \texttt{unknown}
представляется как \texttt{null}. На практике, популярных СУБД эти два значения -- одно и
то же.

\paragraph{Прочие спецэффекты}

\begin{itemize}
	\item \texttt{exists}. Возвращает только \texttt{true} или \texttt{false}. Если
	      все строки
	      состоят из \texttt{null} -- результат равен \texttt{false}.
	\item \textbf{Агрегирующие функции}. По умолчанию пропускают \texttt{null}. Если в результате
	      остается пустой набор аргументов, то результат будет равен \texttt{null}. Единственное
	      исключение -- \texttt{count(*)} -- считает число строк.
	\item \texttt{order by}. Можно указать \texttt{nulls first} или \texttt{nulls last} и
	      определить, считать \texttt{null} меньше или больше других значений соответственно.
\end{itemize}
