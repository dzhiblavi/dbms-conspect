\subsection{Развитие баз данных}
\subsubsection{Простые и структурированные файлы}

Простые файлы состоят из:
\begin{itemize}
	\item Заголовок -- название столбцов.
	\item Данные -- значения, разделённые запятой.
\end{itemize}
В структурированных файлах в заголовках написано не только название столбца, но и его тип и длина.

\begin{remark}
	В структурированной версии можно быстро искать запись по номеру, то есть прочитать заголовок и
	узнать сколько занимает одна запись,умножить на нужный номер и прочесть сразу нужную запись.
\end{remark}

Достоинства:
\begin{itemize}
	\item Простота чтения -- написать код который будет читать такие данные просто.
\end{itemize}

\begin{itemize}
	\item Сложность поиска -- не реализовать эффективный поиск которому не нужно было бы загружать всё в
	      память.
	\item Сложность обработки.
	\item Сложно хранить нетривиальные типы данных например даты - теряется информация на какой позиции
	      месяц, на какой день.
	\item Нет проверки целостности	(ограничений).
\end{itemize}

\subsubsection{Файловая система}

Устройство:
\begin{itemize}
	\item Файл -- одна запись.
	\item Иерархия записей -- иерарзия каталогов.
\end{itemize}

Достоинства:
\begin{itemize}
	\item Простота реализации.
	\item Структурированные данные.
\end{itemize}

Недостатки:
\begin{itemize}
	\item Сложно извлекать требуемые данные.
	\item Нет проверки целостности.
	\item Большое количество файлов.
\end{itemize}

\subsubsection{Иерархическая модель данных}

\begin{remark}
	Иерархия это хорошо, но использовать для этого файловую систему не эффективно.
\end{remark}

\paragraph{Деревья}
Отношение родитель -- ребёнок соответствует каталогу и его подкаталогам в файловой системе,  но не
будет выделяться по файлу для каждой записи, вместо этого записи с одинаковым типом будут
группироваться (благодаря этому не нужно будет лишний раз обходить файловую систему).

Достоинства:
\begin{itemize}
	\item Проверка целостности появляется благодаря стуктурированности (а именно связи родитель - ребёнок),
	      например можно проверять что у человека нет двух оценок по одному предмету (хотя в файловой системе
	      тоже можно было это делать).
	\item Последовательное расположение записей - ускорение выполнения запросов.
\end{itemize}

Недостатки:
\begin{itemize}
	\item Представление только древовидных структур.
	\item Нет отношения многие ко многим, например у множества студентов есть множество оценок по разным
	      предметам и родителем будет студен, а детьми оценки или наоборот, запросы к обоим этим множествам
	      выполниться эффективно не могут.
\end{itemize}

\subsubsection{Сетевая модель данных}

Обобщение иерархических баз данных, нет единой строгой иерархии, есть базовая иерархия и есть
дополнительные иерархии вида владелец -- запись

\enewline

Достоинства:
\begin{itemize}
	\item Представление всех типов связей (в том числе многие-ко-многим).
	\item Возможность описания структуры.
	\item Эффективность реализации -- эффективные запросы к обоим мн-вам из связи многие ко многим, но
	      эффетивность разная из-за последовательной записи, только записи базовой иерархии записаны
	      последовательно.
\end{itemize}

Недостатки:
\begin{itemize}
	\item Более сложная реализация.
	\item Жесткое ограничение структуры -- если мы не подумали о каком то виде запросов заранее, то возможно
	      для его исполнения придётся поднять все данные.
\end{itemize}

\subsubsection{Реляционная модель данных}
\paragraph{Хранение}

Данные хранятся в таблицах, также в таблицах хранится информация о связях, связи задаются в
запросах.

\enewline

Достоинства:
\begin{itemize}
	\item Представление всех типов связей-
	\item Гибкая структура данных -- можно задавать произвольные запросы.
	\item Математическая модель -- позволяет говорить что некоторые запросы эквивалентны, то есть запрос не
	      обязан исполняться как написан, мб исполнен любой эквивалентный запрос, выбирается самый
	      эффективный из эквивалентных и получается тот же самый результат что и при исходном запросе потому
	      что запрсы эквивалентны.
\end{itemize}

Недостатки:
\begin{itemize}
	\item Сложность реализации.
	\item Сложность представления иерархических данных.
	\item Сложность составления эффективных запросов.
\end{itemize}

\subsubsection{Объектные базы данных}
Цель -- хранить граф объектов, который уже находится в памяти, в базе данных. Обычная реализация --
слой трансляции в реляционную базу данных

\enewline

Достоинства:
\begin{itemize}
	\item Работа в терминах объектов а не записей.
	\item Логичное направление ссылок, например можем легко взять все оценки студента потому что есть
	      соответствующее отображение из студента в оценки.
\end{itemize}

Недостатки:
\begin{itemize}
	\item Сложность реализации.
	\item Сложность миграции схемы, например добавление поля объекту, в базе уже есть объекты без этого поля.
	\item Малая распространенность.
\end{itemize}

\subsubsection{NoSQL (Not only SQL)}

\paragraph{Основная мысль} Реляционные базы данных умеют слишком много -- они заточенны чтобы работать
одинаково эффективно в куче различных сценариев, а если у нас какой то один сценарий, то можно
оптимизировать ровно для него и написать эффективней.

\enewline

Различные типы:
\begin{itemize}
	\item Документ-ориентированне -- есть куча документов, важно что внутри них, главное уметь их быстро
	      искать.
	\item Ключ-значение -- всё что предоставляет движок - быстро по ключу достать значение.
	\item Табличные и столбчатые -- хранить таблицы по стобцам, так если у нас множество запросов к
	      конкретным двум столбцам, то мы сможем прочитать только их, читать придётся дважды (каждый столбец
	      отдельно читается), но зато не нужно читать все столбцы как в табличном подходе.
	\item Графовые -- хотим хранить графы.
\end{itemize}

Достоинства:
\begin{itemize}
	\item Большой выбор -- отказываемся почти от всего кроме одного, у чего получаем большую
	      производительность.
	\item Гибкость -- в момент разработки базы, не тогда когда уже есть база.
	\item Скорость работы.
\end{itemize}

Недостатки:
\begin{itemize}
	\item Множество вещей делается в коде.
	\item Нет стандартных оптимизаторов.
	\item Легко ошибиться.
\end{itemize}
