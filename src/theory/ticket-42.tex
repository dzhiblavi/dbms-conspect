\subsection{Трактовки null и операции с ним}

\subsubsection{Трактовки null}

SQL допускает использования в качестве значения \texttt{null} по умолчанию. Однако, это
значение может трактоваться по-разному.

Рассмотрим для этого следующий пример: пусть в БД ``Университет'' идентификатор группы студента
равен \texttt{null}, то есть значение отсутствует. Это может означать одно следующего:

\begin{itemize}
	\item \textbf{Значение не известно}. Неизвестно, в какой группе обучается студент;
	\item \textbf{Значение не верно}. Студент учится в группе, имя которой неверно;
	\item \textbf{Значение еще не существует}. Студент зачислен, но не определен в группу;
	\item \textbf{Значение уже не существует}. Группа отчисленного студента;
	\item \textbf{Значение не имеет смысла}. Студент из другого университета;
	\item \textbf{Значение не доступно}. У пользователя нет права узнать группу.
\end{itemize}

Заметим, что отсутствующие значения можно реализовать без \texttt{null} -- используя
необязательную связь 1:1. Однако, \texttt{null} может появиться в результате
множественных операций и внешних соединений. Поэтому полностью \texttt{null} исключить
нельзя.

\subsubsection{Операции с null}

\paragraph{Тернарная логика}

С точки зрения SQL \textit{результат} результат логического выражения может быть
\texttt{true}, \texttt{false} или \texttt{unknown}. Однако, тип
\texttt{boolean} может \textit{принимать значения} \texttt{true},
\texttt{false} или \texttt{null}. С практической точки зрения,
\texttt{unknown} и \texttt{null} не отличаются.

Рассмотрим основные логические опреации.

\begin{center}
	\begin{tabular}{|c||c|c|c|}
		\hline
		\textbf{AND}     & \texttt{true}    & \texttt{unknown} & \texttt{false} \\
		\hline
		\hline
		\texttt{true}    & \texttt{true}    & \texttt{unknown} & \texttt{false} \\
		\hline
		\texttt{unknown} & \texttt{unknown} & \texttt{unknown} & \texttt{false} \\
		\hline
		\texttt{false}   & \texttt{false}   & \texttt{false}   & \texttt{false} \\
		\hline
	\end{tabular}
\end{center}

\begin{center}
	\begin{tabular}{|c||c|c|c|}
		\hline
		\textbf{OR}      & \texttt{true} & \texttt{unknown} & \texttt{false}   \\
		\hline
		\hline
		\texttt{true}    & \texttt{true} & \texttt{true}    & \texttt{true}    \\
		\hline
		\texttt{unknown} & \texttt{true} & \texttt{unknown} & \texttt{unknown} \\
		\hline
		\texttt{false}   & \texttt{true} & \texttt{unknown} & \texttt{false}   \\
		\hline
	\end{tabular}
\end{center}

\begin{center}
	\begin{tabular}{|c||c|c|c|}
		\hline
		\textbf{NOT} & \texttt{true}  & \texttt{unknown} & \texttt{false} \\
		\hline
		\hline
		             & \texttt{false} & \texttt{unknown} & \texttt{true}  \\
		\hline
	\end{tabular}
\end{center}

Интуитивно, \texttt{true} > \texttt{unknown} > \texttt{false}.
\textbf{AND} возвращает меньшее из аргументов, \textbf{OR} -- большее из
аргументов, \textbf{NOT} -- обратное по порядку.

Для сравнения используются два оператора: бинарный \textbf{=} и унарный
\textbf{is [not] (true | unknown | false)}.

\begin{center}
	\begin{tabular}{|c||c|c|c|}
		\hline
		\textbf{=}       & \texttt{true}    & \texttt{unknown} & \texttt{false}   \\
		\hline
		\hline
		\texttt{true}    & \texttt{true}    & \texttt{unknown} & \texttt{false}   \\
		\hline
		\texttt{unknown} & \texttt{unknown} & \texttt{unknown} & \texttt{unknown} \\
		\hline
		\texttt{false}   & \texttt{false}   & \texttt{unknown} & \texttt{true}    \\
		\hline
	\end{tabular}
\end{center}

\begin{center}
	\begin{tabular}{|c||c|c|c|}
		\hline
		\textbf{is}             & \texttt{true}  & \texttt{unknown} & \texttt{false} \\
		\hline
		\hline
		\textbf{is true}        & \texttt{true}  & \texttt{false}   & \texttt{false} \\
		\hline
		\textbf{is unknown}     & \texttt{false} & \texttt{true}    & \texttt{false} \\
		\hline
		\textbf{is false}       & \texttt{false} & \texttt{false}   & \texttt{true}  \\
		\hline
		\textbf{is not true}    & \texttt{false} & \texttt{true}    & \texttt{true}  \\
		\hline
		\textbf{is not unknown} & \texttt{true}  & \texttt{false}   & \texttt{true}  \\
		\hline
		\textbf{is not false}   & \texttt{true}  & \texttt{true}    & \texttt{false} \\
		\hline
	\end{tabular}
\end{center}

\paragraph{Пример}

Рассмотрим импликацию: через \textbf{OR} (\texttt{A $\rightarrow$ B = (not A) or B}) и через \textbf{AND} (\texttt{A $\rightarrow$ B = not (A and not B)}).

\begin{center}
	\begin{tabular}{|c||c|c|c|}
		\hline
		$\xrightarrow{\textbf{OR}}$ & \texttt{true} & \texttt{unknown} & \texttt{false}   \\
		\hline
		\hline
		\texttt{true}               & \texttt{true} & \texttt{unknown} & \texttt{false}   \\
		\hline
		\texttt{unknown}            & \texttt{true} & \texttt{unknown} & \texttt{unknown} \\
		\hline
		\texttt{false}              & \texttt{true} & \texttt{unknown} & \texttt{true}    \\
		\hline
	\end{tabular}
\end{center}

\begin{center}
	\begin{tabular}{|c||c|c|c|}
		\hline
		$\xrightarrow{\textbf{AND}}$ & \texttt{true} & \texttt{unknown} & \texttt{false}   \\
		\hline
		\hline
		\texttt{true}                & \texttt{true} & \texttt{unknown} & \texttt{false}   \\
		\hline
		\texttt{unknown}             & \texttt{true} & \texttt{unknown} & \texttt{unknown} \\
		\hline
		\texttt{false}               & \texttt{true} & \texttt{true}    & \texttt{true}    \\
		\hline
	\end{tabular}
\end{center}

\begin{proposition}
	Законы Де Моргана неприменимы в тернарной логике.
\end{proposition}

\paragraph{Операции с null}

\texttt{null} заразен. Следующие операции при неизвестном значении хотя бы одного
аргумента возвращают \texttt{null}:

\begin{itemize}
	\item \textbf{Операции сравнения}: \texttt{=}, \texttt{<>}, \texttt{<},
	      \texttt{<=}, \texttt{>}, \texttt{>=};
	\item \textbf{Арифметические операции}: \texttt{+}, \texttt{-}, \texttt{*},
	      \texttt{/};
	\item \textbf{Конкатенация строк}: \texttt{||};
	\item \textbf{Подзапрос вхождения}: \texttt{in}:
	      \begin{itemize}
		      \item \texttt{in(true, false, null) == null}.
	      \end{itemize}
\end{itemize}

Для избавления от \texttt{null} используется операция \texttt{coalesce(v1, v2, ...)},
принимающая произвольное число аргументов и возвращающая первый из них, не равный
\texttt{null}, или \texttt{null} при отсутствии таковых.

\paragraph{Дубликаты}

\texttt{null != null}, следовательно, одинаковые кортежи, содержащие \texttt{null}, не
равны: \[
	(1, \texttt{null}) \neq (1, \texttt{null}).
\]

Пусть в отношении $R$ содержится такой кортеж. Тогда:
\begin{align}
	R \cup R \neq R \\
	R \cap R \neq R \\
	R \bowtie R \neq R
\end{align}

При объединении такие кортежи будут продублированы, при пересечении и естественном соединении --
удалены.

\paragraph{Спецэффекты}

\begin{itemize}
	\item \texttt{x = x}. Результат -- \texttt{true} или \texttt{null};
	\item \texttt{x <> x}. Результат -- \texttt{true} или \texttt{null};
	\item \texttt{x or x}. Результат -- \texttt{x}
	\item \texttt{x or not x}. Результат -- \texttt{true} или \texttt{null};
	\item \texttt{x and not x}. Результат -- \texttt{false} или \texttt{null};
	\item \texttt{x or not x}. Результат -- \texttt{true} или \texttt{null};
	\item \textbf{Отсутствуют транзитивность и рефлексивность сравнения}.
\end{itemize}

\paragraph{Ключи}

\begin{itemize}
	\item \textbf{Основной ключ}. Не могут содержать \texttt{null} по стандарту.
	\item \textbf{Дополнительный ключ}. Использование \texttt{null} разрешено.
	\item \textbf{Простой внешний ключ}. \texttt{null} разрешен -- такое значение трактуется
	      как отсутствующая ссылка, по ней не будет проходить соединение.
	\item \textbf{Составной внешний ключ}. Поддерживаются внешние ключи, отсутствующие целиком.
	      Если отсутствует только часть ключа, то <<ничем хорошим это не закончится>>.
\end{itemize}

\paragraph{Интуитивность}

Рассмотрим пример. Ожидается, что запрос возвращает информацию о студентах, не учащихся в
определенной группе:

\begin{lstlisting}[language=SQL]
    select * from Students where GId <> 'M34391'
\end{lstlisting}

Пусть поле \texttt{GId} может отсутствовать. Корректность запроса зависит от смысла
отсутствующего значения. Если смысл -- ``значение не известно'', то неизвестно, должен ли такой
кортеж попасть в результат. Если же ``значение неприменимо'' (например, студент другого
университета) -- то такой кортеж точно не должен быть в ответе.

Рассмотрим другой пример. Ожидается, что запрос возвращает информацию о всех студентах:

\begin{lstlisting}[language=SQL]
    select * from Students where GId <> 'M34391'
    union
    select * from Students where GId = 'M34391'
\end{lstlisting}

В действительности будет получена информация по всем студентам, у которых группа
\textit{не отсутствует}.

\paragraph{Вывод}

При работе с данными, значения которых могут отсутствовать, нужно как учитывать спецэффекты
тернарной логики и операторов при работе с \texttt{null}, так и смысл, который
вкладывается в отсутствующее значение.
