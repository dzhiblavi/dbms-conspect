\subsection{Трактовки null и операции с ним}

\subsubsection{Трактовки null}

SQL допускает использования в качестве значения \texttt{null} по умолчанию. Однако, это
значение может трактоваться по-разному.

Рассмотрим для этого следующий пример: пусть в БД ``Университет'' идентификатор группы студента
равен \texttt{null}, то есть значение отсутствует. Это может означать одно следующего:

\begin{itemize}
	\item \textbf{Значение не известно}. Неизвестно, в какой группе обучается студент;
	\item \textbf{Значение не верно}. Студент учится в группе, имя которой неверно;
	\item \textbf{Значение еще не существует}. Студент зачислен, но не определен в группу;
	\item \textbf{Значение уже не существует}. Группа отчисленного студента;
	\item \textbf{Значение не имеет смысла}. Студент из другого университета;
	\item \textbf{Значение не доступно}. У пользователя нет права узнать группу.
\end{itemize}

Заметим, что отсутствующие значения можно реализовать без \texttt{null} -- используя
необязательную связь 1:1. Однако, \texttt{null} может появиться в результате
множественных операций и внешних соединений. Поэтому полностью \texttt{null} исключить
нельзя.

\subsubsection{Операции с null}

\paragraph{Тернарная логика}

С точки зрения SQL \textit{результат} результат логического выражения может быть
\texttt{true}, \texttt{false} или \texttt{unknown}. Однако, тип
\texttt{boolean} может \textit{принимать значения} \texttt{true},
\texttt{false} или \texttt{null}. С практической точки зрения,
\texttt{unknown} и \texttt{null} не отличаются.

Рассмотрим основные логические опреации.

\begin{center}
	\begin{tabular}{|c||c|c|c|}
		\hline
		\textbf{AND}     & \texttt{true}    & \texttt{unknown} & \texttt{false} \\
		\hline
		\hline
		\texttt{true}    & \texttt{true}    & \texttt{unknown} & \texttt{false} \\
		\hline
		\texttt{unknown} & \texttt{unknown} & \texttt{unknown} & \texttt{false} \\
		\hline
		\texttt{false}   & \texttt{false}   & \texttt{false}   & \texttt{false} \\
		\hline
	\end{tabular}
\end{center}

\begin{center}
	\begin{tabular}{|c||c|c|c|}
		\hline
		\textbf{OR}      & \texttt{true} & \texttt{unknown} & \texttt{false}   \\
		\hline
		\hline
		\texttt{true}    & \texttt{true} & \texttt{true}    & \texttt{true}    \\
		\hline
		\texttt{unknown} & \texttt{true} & \texttt{unknown} & \texttt{unknown} \\
		\hline
		\texttt{false}   & \texttt{true} & \texttt{unknown} & \texttt{false}   \\
		\hline
	\end{tabular}
\end{center}

\begin{center}
	\begin{tabular}{|c||c|c|c|}
		\hline
		\textbf{NOT} & \texttt{true}  & \texttt{unknown} & \texttt{false} \\
		\hline
		\hline
		             & \texttt{false} & \texttt{unknown} & \texttt{true}  \\
		\hline
	\end{tabular}
\end{center}

Интуитивно, \texttt{true} > \texttt{unknown} > \texttt{false}.
\textbf{AND} возвращает меньшее из аргументов, \textbf{OR} -- большее из
аргументов, \textbf{NOT} -- обратное по порядку.

Для сравнения используются два оператора: бинарный \textbf{=} и унарный
\textbf{is [not] (true | unknown | false)}.

\begin{center}
	\begin{tabular}{|c||c|c|c|}
		\hline
		\textbf{=}       & \texttt{true}    & \texttt{unknown} & \texttt{false}   \\
		\hline
		\hline
		\texttt{true}    & \texttt{true}    & \texttt{unknown} & \texttt{false}   \\
		\hline
		\texttt{unknown} & \texttt{unknown} & \texttt{unknown} & \texttt{unknown} \\
		\hline
		\texttt{false}   & \texttt{false}   & \texttt{unknown} & \texttt{true}    \\
		\hline
	\end{tabular}
\end{center}

\begin{center}
	\begin{tabular}{|c||c|c|c|}
		\hline
		\textbf{is}             & \texttt{true}  & \texttt{unknown} & \texttt{false} \\
		\hline
		\hline
		\textbf{is true}        & \texttt{true}  & \texttt{false}   & \texttt{false} \\
		\hline
		\textbf{is unknown}     & \texttt{false} & \texttt{true}    & \texttt{false} \\
		\hline
		\textbf{is false}       & \texttt{false} & \texttt{false}   & \texttt{true}  \\
		\hline
		\textbf{is not true}    & \texttt{false} & \texttt{true}    & \texttt{true}  \\
		\hline
		\textbf{is not unknown} & \texttt{true}  & \texttt{false}   & \texttt{true}  \\
		\hline
		\textbf{is not false}   & \texttt{true}  & \texttt{true}    & \texttt{false} \\
		\hline
	\end{tabular}
\end{center}

\paragraph{Пример}

Рассмотрим импликацию: через \textbf{OR} (\texttt{A $\rightarrow$ B = (not A) or B}) и через \textbf{AND} (\texttt{A $\rightarrow$ B = not (A and not B)}).

\begin{center}
	\begin{tabular}{|c||c|c|c|}
		\hline
		$\xrightarrow{\textbf{OR}}$ & \texttt{true} & \texttt{unknown} & \texttt{false}   \\
		\hline
		\hline
		\texttt{true}               & \texttt{true} & \texttt{unknown} & \texttt{false}   \\
		\hline
		\texttt{unknown}            & \texttt{true} & \texttt{unknown} & \texttt{unknown} \\
		\hline
		\texttt{false}              & \texttt{true} & \texttt{unknown} & \texttt{true}    \\
		\hline
	\end{tabular}
\end{center}

\begin{center}
	\begin{tabular}{|c||c|c|c|}
		\hline
		$\xrightarrow{\textbf{AND}}$ & \texttt{true} & \texttt{unknown} & \texttt{false}   \\
		\hline
		\hline
		\texttt{true}                & \texttt{true} & \texttt{unknown} & \texttt{false}   \\
		\hline
		\texttt{unknown}             & \texttt{true} & \texttt{unknown} & \texttt{unknown} \\
		\hline
		\texttt{false}               & \texttt{true} & \texttt{true}    & \texttt{true}    \\
		\hline
	\end{tabular}
\end{center}

\begin{proposition}
	Законы Де Моргана неприменимы в тернарной логике.
\end{proposition}
