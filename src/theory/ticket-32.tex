\subsection{Транзакции. Параллельное исполнение. Блокировки.}

\subsubsection{Параллельное исполнение}

Напомним, что свойство \textit{изоляции (isolation)} транзакции подразумевает, что она должна
исполняться так, как будто она в системе одна, а также корректные по отдельности транзакции должны
быть корректными в совокупности. Следовательно, транзакции могут исполняться параллельно.

Обычно параллельные транзакции выполняются в разных потоках, что приводит к необходимости
использования блокировок для синхронизации.

\paragraph{Примеры конфликтов}

\begin{itemize}
    \item \textbf{Потерянное обновление (1)}. Обновление, сделанное транзакцией 1, потеряно.
        \begin{center}
            \begin{tabular}{|c c|}
                \hline
                Транзакция 1 & Транзакция 2 \\
                \hline
                retrieve T & \\
                & retrieve T \\
                update T & \\
                & update T \\
                commit & \\
                & commit \\
                \hline
            \end{tabular}
        \end{center}
    \item \textbf{Потерянное обновление (2)}. Обновление, сделанное транзакцией 1, потеряно.
        \begin{center}
            \begin{tabular}{|c c|}
                \hline
                Транзакция 1 & Транзакция 2 \\
                \hline
                & update T \\
                update T & \\
                commit & \\
                & rollback \\
                \hline
            \end{tabular}
        \end{center}
    \item \textbf{Незафиксированное изменение}. Значение, полученное транзакцией 1, не было
        зафиксировано.
        \begin{center}
            \begin{tabular}{|c c|}
                \hline
                Транзакция 1 & Транзакция 2 \\
                \hline
                & update T \\
                retrieve T & \\
                & rollback \\
                commit & \\
                \hline
            \end{tabular}
        \end{center}
    \item \textbf{Несогласованное состояние}. Значение, полученное транзакцией 1, не могло быть
        получено в согласованном состоянии.
        \begin{center}
            \begin{tabular}{|c c|}
                \hline
                Транзакция 1 & Транзакция 2 \\
                \hline
                & update T\_1 = T\_1 - 10 \\
                retrieve T\_1 & \\
                retrieve T\_2 & \\
                & update T\_2 = T\_2 + 10 \\
                commit & \\
                & commit \\
                \hline
            \end{tabular}
        \end{center}
\end{itemize}

\paragraph{Типы конфликтов}

\begin{itemize}
    \item \textbf{Чтение-чтение}. Нет конфликтов.
    \item \textbf{Чтение-запись}. Некорректное состояние.
    \item \textbf{Запись-чтение}. Зависимость от незафиксированного изменения.
    \item \textbf{Запись-запись}. Потерянное обновление.
\end{itemize}

\subsubsection{Блокировки}

Нам потребуются многоуровневые блокировки на фрагменты данных: разделяемая (для чтения, \textit{S})
и эксклюзивная (для записи, \textit{X}). Отсутствие блокировки обозначим \textit{-}.

\begin{definition}
    \textit{Протокол двухфазной блокировки}. Для чтения требуется получение \textit{S}, для
    записи -- \textit{X}, при завершении или откате транзакции требуется
    \textbf{последовательно} освободить все блокировки. Важно, что в первую фазу количество
    блокировок только растет, а во вторую -- только уменьшается.
\end{definition}

\begin{definition}
    \textit{Строгий протокол двухфазной блокировки}. Аналогично предыдущему определению,
    но все блокировки во второй фазе необходимо отпускать \textbf{после завершения или отката}
    транзакции.
\end{definition}

\paragraph{Взаимная блокировка (ВБ)}

\textbf{Пример взаимной блокировки}

\begin{center}
    \begin{tabular}{|c c|}
        \hline
        Транзакция 1 & Транзакция 2 \\
        \hline
        retrieve T\_1 & \\
        & retrieve T\_2 \\
        update T\_2 & \\
        & update T\_1 \\
        \hline
    \end{tabular}
\end{center}

Транзакция 1 и транзакция 2 сначала берут \textit{S} блокировки на чтение \textit{T\_1} и
\textit{T\_2} соответственно. Затем транзакция 1 пытается взять \textit{X} блокировку на запись
в \textit{T\_2}, что не удается сделать, пока транзакция 2 владеет \textit{S} блокировкой
\textit{T\_2}. Аналогично, транзакция 2 не может взять \textit{X} блокировку на запись, поскольку
так как транзакция 1 владеет \textit{S} блокировкой на \textit{T\_2}.

\paragraph{Устранение ВБ}

\begin{itemize}
    \item \textbf{Построение графа ожиданий}. Наличие цикла в таком графе свидетельствует о ВБ. На
        практике графы слишком большие, поэтому данный подход менее популярен.
    \item \textbf{Выставление таймаутов}. Отсутствие прогресса на протяжении долгого времени
        вероятно свидетельствует о ВБ.
\end{itemize}

При обнаружении ВБ следует откатить одну из транзакций. Также, если СУБД владеет транзакцией, то
есть может ее перезапустить, то это следует сделать.

\paragraph{Предотвращение ВБ}

Пусть транзакция \textit{A} претендует на блокировку, конфликтующую с блокировками транзакции
\textit{B}. Возможны следующие стратегии.

\begin{itemize}
    \item \textbf{Стратегия ожидание-отмена}.
        \begin{itemize}
            \item \textit{A} началась раньше \textit{B} -- \textit{A} ожидает;
            \item \textit{A} началась позже \textit{B} -- \textit{A} отменяется (и, по
                возможности, перезапускается);
        \end{itemize}
    \item \textbf{Стратегия отмена-ожидание}.
        \begin{itemize}
            \item \textit{A} началась раньше \textit{B} -- \textit{B} отменяется (и, по
                возможности, перезапускается);
            \item \textit{A} началась позже \textit{B} -- \textit{A} ожидает;
        \end{itemize}
\end{itemize}

ВБ в каждой стратегии исключается, поскольку в графе ожидания ребра идут только от старшей к
младшей или от младшей к старшей транзакциям соответственно.

Стоит отметить, что стратегии порождают много лишних откатов.

\paragraph{Упорядочиваемость}

\begin{definition}
    \textit{Упорядочиваемость (serializability)} -- любая последовательность исполнения транзакций эквивалентна
    (равны состояния до начала и после окончания исполнения) некоторому последовательному
    исполнению.
\end{definition}

\begin{proposition}
    \textbf{Строгий} протокол двухфазной блокировки гарантирует упорядочиваемость.
\end{proposition}

\begin{proposition}
    Протокол двухфазной блокировки гарантирует упорядочиваемость.
\end{proposition}

\subsubsection{Восстановление и параллелизм}

Рассмотрим следующий пример.

\begin{center}
    \begin{tabular}{|c c|}
        \hline
        Транзакция 1 & Транзакция 2 \\
        \hline
        & update T \\
        retrieve T & \\
        commit & \\
        & rollback \\
        \hline
    \end{tabular}
\end{center}

Транзакция 1 фиксирует изменения, внесенные транзакцией 2. Однако, в будущем транзакция 2 может
быть откачена, например, из-за сбоя. Таким образом, фиксируется изменение, зависящее от
незафиксированного.

\begin{definition}
    \textit{Критерий восстанавливаемости}. Если транзакция \textit{A} использует значения,
    обновленные транзакцией \textit{B}, то \textit{A} должна завершиться позже, чем \textit{B}.
\end{definition}

В случае противоречий возникают взаимные блокировки, способы борьбы с которыми были рассмотрены
выше. Однако, это может привести к цепочкам форсированных откатов (откат транзакции \textit{B}
форсирует откат транзакции \textit{A}).

\begin{proposition}
    При строгом протоколе двухфазной блокировки цепочки отката отсутствуют.
\end{proposition}

\begin{proof}
    Транзакция \textit{A} сможет получить блокировку на чтение только
    после отпускания эксклюзивной блокировки на запись транзакцией \textit{B}. Последнее в строгом
    протоколе двухфазной блокировки произойдет только после завершения транзакции \textit{B}.
\end{proof}

\subsubsection{Гранулярность блокировок}

\begin{itemize}
    \item \textbf{Блокировка поля записи}. Блокируются отдельные поля каждой записи. Не
        используется на практике.
    \item \textbf{Блокировка записей}. Блокируются отдельные записи. Дает высокий параллелизм и
        требует больших ресурсов.
    \item \textbf{Блокировка страниц}. Блокируется страница памяти, на которой расположена запись.
        Более практично по сравнению с блокировкой отдельных записей, поскольку на одной странице
        может быть расположено много записей.
    \item \textbf{Блокировка индексов}. Блокируется элемент (например, поддерево в B-дереве или
        корзина в хеш-индексе) или страница индекса. Запрещает добавление или удаление в рамках
          заблокированного индекса.
    \item \textbf{Блокировка таблиц}. Блокируется таблица целиком. Требует мало ресурсов и
        предоставляет низкий параллелизм.
    \item \textbf{Блокировка БД}. Используется для резервного копирования, изменения определения
        таблиц и представлений, изменения хранимых процедур и функций и изменения прав доступа.
\end{itemize}

Аномалия \textbf{фантомные записи} -- при повторном чтении могут появиться новые записи. Возможна
при гранулярности блокировки меньше, чем по таблицам.
