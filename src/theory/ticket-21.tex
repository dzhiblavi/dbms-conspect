\subsubsection{Целостность данных. Триггеры}

\subsubsection{CRUD}

Базы данных реализуют стандартный интерфейс \textit{CRUD}:

\begin{itemize}
	\item \textit{Create} -- оператор \texttt{insert},
	\item \textit{Read} -- оператор \texttt{select},
	\item \textit{Update} -- оператор \texttt{update},
	\item \textit{Delete} -- оператор \texttt{delete}.
\end{itemize}

\paragraph{Insert}

Оператор вставки (был рассмотрен ранее). В случае, если список атрибутов (\textit{attributes})
опущен, ожидаются все в порядке, в котором они были определены в DDL. Не все СУБД поддерживают
вставку нескольких строк в рамках одного запроса.

\begin{lstlisting}[language=SQL]
    insert into <table> [(<attributes>)]
    values (<values>) [, <more lines>];
    insert into <table> [(<attribues>)]
    <query>;
\end{lstlisting}

\begin{examples}
	\enewline
	\begin{lstlisting}[language=SQL]
    insert into Courses (Name, Lecturer) values
    ('DBMS', 'John Smith'),
    ('Software Design', 'Smith Johnson');
    insert into Points
    select SId, CId, 0 from Students, Courses;
    \end{lstlisting}
\end{examples}

\paragraph{Update}

Оператор обновления. Часть \texttt{where} является необязательной, и при отсутствии
обновляются все строки.

\begin{lstlisting}[language=SQL]
    update <table> set
    <attribute> = <value> [, <attribute> = <value>, ...]
    [where <condition>]
\end{lstlisting}

\begin{examples}
	\enewline
	\begin{lstlisting}[language=SQL]
    update Students set GroupId = '4537'
    where LastName = 'Smith';
    update Students set
    PassNo = '1234', PassSeries = '567890'
    where SId = 1;
    update Points set points = points + 1 where CId in
    (select CId from Courses where Name = 'DBMS');
    \end{lstlisting}
\end{examples}

\paragraph{Delete}

Оператор удаления. Аналогично \texttt{Update}, часть \texttt{where} является
необязательной, и при отсутствии удаляются все строки. Другой способ полностью очистить таблицу --
оператор \texttt{truncate}, формально относящийся к DDL.

\begin{lstlisting}[language=SQL]
    delete from <table> [where <condition>];
    truncate <table>;
\end{lstlisting}

\begin{examples}
	\enewline
	\begin{lstlisting}[language=SQL]
    delete from Points where CId in
    (select CId from Courses where Name = 'DBMS);
    truncate Points;
    \end{lstlisting}
\end{examples}

\paragraph{Merge}

Оператор слияния. Объединяет вставку и обновление. Поддерживается не всеми СУБД. Дополнительные
условия (\texttt{match condition}, \texttt{not match condition}) также поддерживаются не всеми СУБД. Также
некоторые СУБД при поддержке дополнительных условий позволяют сделать несколько веток на
\texttt{when matched} и \texttt{when not matched}. Действием (\texttt{action}) может быть
\texttt{insert}, \texttt{update} или \texttt{delete} (но не все из них имеют смысл, например,
\texttt{update} или \texttt{delete} в ветке \texttt{not matched}).

\begin{lstlisting}[language=SQL]
    merge into <table> [[as] <alias 1>]
    using <data> [[as] <alias 2]
    on <condition>
    when matched [and <match condition>] then <action 1>
    when not matched [and <not match condition>] then <action 2>
\end{lstlisting}

\begin{example}
	\enewline
	\begin{lstlisting}[language=SQL]
    merge into Points
    using (select * from Courses c, Students s) r
    on Points.CId = r.Cid and Points.SId = r.SId
    when matched then update set points = points + 1
    when not matched then insert (SId, CId, points)
        values (r.SId, r.CId, 1)
    \end{lstlisting}
\end{example}

\subsubsection{Типы ограничения целостности данных}

\begin{definition}
	\textit{Корректность данных} -- соответствие содержимого БД ``реальному миру''. Не может быть
	проверена в терминах БД.
\end{definition}

\begin{examples}
	\enewline
	\begin{itemize}
		\item Оценка каждого студента соответствует полученной на экзамене;
		\item Студент слушает курс ``БД''.
	\end{itemize}
\end{examples}

Для БД имеет смысл понятие \textit{целостности}.

\begin{definition}
	\textit{Целостность} -- непротиворечивость содержимого БД. Может быть описана набором правил и
	проверена.
\end{definition}

\begin{examples}
	\enewline
	\begin{itemize}
		\item У студента есть оценки только по предметам, которые слушала его группа;
		\item Студент посетил столько занятий курса, сколько их было проведено.
	\end{itemize}
\end{examples}

По умолчанию проверка целостности происходит при завершении работы \textit{каждого оператора}. Также в
рамках \textit{транзакции} возможно отложить проверки целостности некоторых ограничений. При
обнаружении нарушения целостности все изменения отменяются.

\paragraph{Ограничение типа}

Ограничение задает множество значений данных, допустимых для типа. Некоторые СУБД также
поддерживают перечисления, составные типы данных и объявление ограничений на существующие типы.

\begin{examples}
	\enewline
	\begin{lstlisting}[language=SQL]
    -- Date type
    Date
    -- Custom type of two string of length from 0 to 100
    create type Name as object (
        FirstName varchar(100),
        LastName varchar(100)
    )
    -- Points of limited range
    create type Points as object (
        points as int between 0 and 100
    )
    \end{lstlisting}
\end{examples}

\paragraph{Ограничение атрибута}

Ограничение задает множество значений, допустимых для атрибута. Является следствием его типа.

\begin{examples}
	\enewline
	\begin{lstlisting}[language=SQL]
    -- Date type
    Enroll Date not null
    -- Custom type of two string of length from 0 to 100
    SName Name
    -- Points of limited range
    CoursePoints Points
    \end{lstlisting}
\end{examples}

\paragraph{Ограничение отношения}

Ограничение позволяет задать несколько предикатов, конъюнкция которых будет проверяться. Типы
предикатов: \textit{ключи}, \textit{внешние ключи}, \textit{проверяемые условия} (или
\textit{check constraint}).

\begin{examples}
	\enewline
	\begin{lstlisting}[language=SQL]
    -- Primary key
    primary key (SId, CId)
    -- Key
    unique (PassNo, PassSeries)
    -- Foreign key
    foreign key (SId) references Students(SId)
    -- Check constraint
    check (Points between 0 and 100)
    \end{lstlisting}
\end{examples}

\paragraph{Проверяемые условия}

Проверяются для каждого кортежа, неявный квантор всеобщности $\forall$. Следовательно,
пустые отношения не проверяются на условия. Оптимизатор СУБД может упрощать реализацию проверок,
где это корректно.

\begin{examples}
	\enewline
	\begin{lstlisting}[language=SQL]
    -- Manual foreign key check
    SId in (select SId from Students);
    -- Consistency check
    ExpulsionDate not null or
    not exists (select * from Points p
                where p.SId = SId, p.CId = CId and Point < 60);
    \end{lstlisting}
\end{examples}

\paragraph{Ограничения базы данных}

Задается множество значений, которые может принимать база данных. Например, через предикаты таблиц
или предикаты утверждений.

\begin{example}
	\textit{Предикат утверждений}:
	\begin{lstlisting}[language=SQL]
    create assertion NonEmptyGroup check (not exists
        (select * from Group g where not exists
            (select * from Students s where s.GId = g.GId)))
    \end{lstlisting}
\end{example}

\subsubsection{Компенсирующие действия}

Рассмотрим пример: в БД ``Университет'' удаляется курс. При этом на него могли ссылаться некоторые
оценки. Можно поступить несколькими способами:

\begin{itemize}
	\item Удалить все оценки по курсу. Данный подход называется \textbf{каскадированием изменений} -- изменение
	      распространяется на зависимые данные;
	\item Запретить удаление. Данный подход называется \textbf{ограничением изменений}.
\end{itemize}

Существуют два опасных действия, которые могут затронуть зависимые данные: \textbf{удаление}
(\texttt{delete}) и
\textbf{обновление} (\texttt{update}). В SQL поддерживаются следующие типы компенсирующих действий:

\begin{itemize}
	\item \textbf{Бездействие} (\texttt{no action}). Подразумевается, что в рамках транзакции будут
	      произведены действия, возвращающие БД в согласованное состояние;
	\item \textbf{Каскадирование} (\texttt{cascade}). Изменения будут распространены на все
	      зависимые записи (в том числе не прямо зависимые);
	\item \textbf{Ограничение} (\texttt{restrict}). При наличии хотя бы одной зависимой записи
	      удаление или обновление следует запретить;
	\item \textbf{Обнуление} (\texttt{set null}). Изменить во всех зависимых записях значение на
	      \texttt{null};
	\item \textbf{Установка значения по умолчанию} (\texttt{set default}). Аналогично предыдущему,
	      но устанавливает значение, определенное по умолчанию.
\end{itemize}

\paragraph{Синтаксис}
\enewline

\begin{lstlisting}[language=SQL]
    foreign key (<columns>)
    references <table> [(<columns>)]
    [on delete <action>]
    [on update <action>]
\end{lstlisting}

\begin{examples}
	\enewline
	\begin{lstlisting}[language=SQL]
        ...
        foreign key (CId) references Courses
        on delete cascade on update cascade;
        ...
        foreign key (SId) references Students
        on delete restrict on update cascade;
    \end{lstlisting}
\end{examples}

\begin{remark}
	Действие по умолчанию -- \texttt{no action}. При нарушении целостности транзакции будут
	откачены.
\end{remark}

\subsubsection{Триггеры}

\begin{definition}
	\textit{Триггер} -- действие, выполняемое при изменении данных. Обычно применяются для
	обеспечения целостности данных, но в общем случае могут производить любые действия.
\end{definition}

Возможные применения триггеров:

\begin{itemize}
	\item Уведомление пользователей,
	\item Обновления денормализованных данных,
	\item Аудит изменений,
	\item Компенсирующие действия.
\end{itemize}

\paragraph{Синтаксис}
\enewline

\begin{lstlisting}[language=SQL]
    create trigger <name> on <object>
    {before | after | instead of} {insert | update | delete}
    [<references declaration>]
    [for each {row | statement}]
    <action>

    -- references declaration
    referencing {new | old} {row | table} name
\end{lstlisting}

Объявление ссылок (\texttt{references declaration}) позволяет сослаться на старую или новую таблицу или
строку -- в зависимости от выбранной гранулярности триггера (\texttt{for each \{row | statement\}}).

\begin{examples}
	\enewline
	\begin{lstlisting}[language=SQL]
        -- Delete propagation
        create trigger on Course after delete
        referencing old row c for each row
        delete from points p where p.CId = c.CId;

        -- Updates audit
        create trigger on Points after insert, update
        referencing new row p
        for each row set p.time = now()
    \end{lstlisting}
\end{examples}
