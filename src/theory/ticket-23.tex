\subsection{Управление доступом к данным}

\subsubsection{Схемы управления доступом}

Существуют две популярные схемы организации доступа:

\begin{itemize}
	\item \textbf{Избирательная (дискреционная)}. Основана на списках доступа, где указано, к чему
	      имеет доступ каждый пользователь. В схеме существуют объекты, на каждый из которых существует набор
	      прав, которые друг с другом не связан. Пользователи могут иметь различные права. Также
	      предусмотрены группы пользователей, права групп при этом распространяются на членов группы;
	\item \textbf{Мандатная}. Данным и пользователям присваиваются классификационные уровни. Данные
	      более низкого уровня являются частью данных более высокого уровня. Чтение разрешено при условии,
	      что уровень ресурса \textit{меньше} уровн пользователя. Изменение разрешено при условии
	      \textit{равенства} уровней (иначе изменение данных более низкого уровня на основе данных более
	      высокого уровня может привести к их рассекречиванию). Добавление записей разрешено на уровень
	      \textit{больший или равный} уровню пользователя.
\end{itemize}

\subsubsection{Пользователи и группы}

По стандарту в SQL применяется \textbf{дискреционная} политика организации доступа, существуют
пользователи и группы (однако, не существует стандартного способа управления ими).

\paragraph{Синтаксис}
\enewline

\begin{lstlisting}[language=SQL]
    create user <name> [password <password>];
    alter user <name> [password <password>];
    drop user <name>;
    create group <name>;
    alter group <name> {add | drop} user <username>
    drop group <name>;
\end{lstlisting}

\begin{remark}
	Большинство СУБД поддерживают синтаксис, схожий с приведенным выше. При этом он не относится к
	стандарту.
\end{remark}

\subsubsection{DCL (Data Control Language)}

Стандартизированный SQL способ выдачи прав пользователям и группам.

\paragraph{Синтаксис добавления прав}
\enewline

\begin{lstlisting}[language=SQL]
    grant {select | insert | update | delete | create
           | execute | trigger | usage | ... | all privileges}
    on {table | database | view | procedure
        | function | ...} <name>
    to {<username> | group <groupname> | public}
    [with grant option]
\end{lstlisting}

Множество возможных прав зависит от объекта, на который они выдаются. Опция \texttt{with grant option}
дает право передоверия выданных прав.

\begin{examples}
	\enewline
	\begin{lstlisting}[language=SQL]
    grant all privileges on Students
    to group Deans with grant option;
    grant select on Students to public;
    \end{lstlisting}
\end{examples}

\paragraph{Синтаксис удаления прав}
\enewline

\begin{lstlisting}[language=SQL]
    revoke [grant option for]
    {select | insert | update | delete | create
     | execute | trigger | usage | ... | all privileges}
    on {table | database | view | procedure
        | function | ...} <name>
    from {<username | group <groupname> | public}
    [cascade | restrict]
\end{lstlisting}

\texttt{cascade} и \texttt{restrict} определяют, нужно ли забирать права у всех, кому
были передоверены права, или следует ли в таком случае запретить изменение прав. Использование
\texttt{revoke grand option for}, удаляет только право передоверия, но не само право.

\begin{examples}
	\enewline
	\begin{lstlisting}[language=SQL]
    revoke insert on Students
    from group Deans cascade;
    revoke update on Students from public;
    \end{lstlisting}
\end{examples}

\paragraph{Владелец объекта}

Пользователь, создавший объект, является его владельцем, и не может ограничить свои права. При этом
владельца можно изменить.

\paragraph{Синтаксис}
\enewline

\begin{lstlisting}[language=SQL]
    alter {table | schema | database | ...} <name>
    owner to <user>
\end{lstlisting}

\subsubsection{Права и представления}

\paragraph{Фильтрующая таблица}

Можно создать представление, фильтрующие данные, и выдать права на него. Таким образом,
пользователь, группа или все получат доступ к ограниченному набору данных из таблицы. Напомним, что
такие представления изменяемые.

\begin{example}
	\enewline
	\begin{lstlisting}[language=SQL]
    create view FITPStudents as
    select * from students
    where GId in
    (select GId from Groups where FId = 'ITP');

    grant all privileges on FITPStudents to FITPDean;
    \end{lstlisting}
\end{example}

\paragraph{Спроецированная таблица}

Аналогично фильтрующей таблице, спроецированная дает доступ к ограниченному набору столбцов. Такие
представления также изменяемые.

\begin{example}
	\enewline
	\begin{lstlisting}[language=SQL]
    -- e.g. no passport info
    create view PublicStudents as
    select SId, FirstName, LastName from students;

    grant select privileges on PublicStudents to public;
    \end{lstlisting}
\end{example}

\paragraph{Агрегированная таблица}

Также можно выдать более широкий доступ на таблицу, которая содержит агрегированные данные. Такие
представления почти ни в какой реализации СУБД не являются изменяемыми.

\begin{example}
	\enewline
	\begin{lstlisting}[language=SQL]
    create view StudentPoints as
    select SId, avg(Points)
    from Students natural join Marks;

    grant select privileges on StudentPoints to public;
    \end{lstlisting}
\end{example}
