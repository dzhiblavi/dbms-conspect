\subsection{Современные РСУБД}
\subsubsection{Корпоративные}

\paragraph{Корпоративные СУБД} предназначены для продажи большим корпорациям, но у большинства
таких РСУБД есть разработческие лицензии, которые позволяют использовать их в ограниченной среде
(ограничение на количество ядер и размер памяти).

\begin{itemize}
	\item Oracle (Oracle)
	      \begin{itemize}
		      \item Высокая пропускная способность (throughput).
		      \item Невысокая скорость обновления (latency).
	      \end{itemize}
	      \begin{remark}
		      Два утверждения выше не противоречат друг другу, хотя каждый запрос и исполняется медленно,
		      пропускная способность получается большой за счёт того, что конкретная СУБД заточена на исполнение
		      тысяч параллельных запросов, и суммарная пропускная способность всех этих запросов, а не отдельных
		      запросов, будет высокой.
	      \end{remark}
	\item DB2 (IBM)
	      \begin{itemize}
		      \item Ориентация на «большие» машины, то есть с точки зрения IBM, СУБД это не приложение, которое
		            крутится на сервере, а отдельное железо.
		      \item Мало распространена в России, так как развивалась в 60'е - 80'е годы предыдущего века.
		      \item Неполная совместимость с SQL.
	      \end{itemize}
	\item SQL Server (Microsoft)
	      \begin{itemize}
		      \item Работа под Windows.
		      \item Масштабируемость (путём добавления новых процессоров).
	      \end{itemize}
\end{itemize}

\subsubsection{Свободные}
\begin{itemize}
	\item MySQL
	      \begin{itemize}
		      \item Поддерживаются различные форматы хранения БД.
		      \item Неполная поддержка SQL.
		      \item Есть enterprise и community версии.
	      \end{itemize}
	\item PostgreSQL
	      \begin{itemize}
		      \item Непосредственная поддержка связей -- СУБД достаточно стабильна, чтобы использовать в реальных
		            проектах.
		      \item Объектные расширения -- но в то же время эта СУБД -- экспериментальный проект, в который
		            добавляется куча различных возможностей, некоторые из которых не выходят из экспериментального
		            статуса.
	      \end{itemize}
	\item Firebird
	      \begin{itemize}
		      \item Была очень популярна когда делалась \href{https://en.wikipedia.org/wiki/Borland}{Borland}'ом под Delphi.
		      \item Используется только в старых проектах, так как в БД, которые используют это СУБД, есть данные,
		            которые нельзя потерять, а перенести их очень сложно.
	      \end{itemize}
\end{itemize}

\subsubsection{Встраиваемые}
\begin{itemize}
	\item SQLite
	      \begin{itemize}
		      \item Компактна, поэтому много используется на мобильных устройствах.
		      \item In-memory mode -- все данные должны поместиться в память.
		      \item Ограниченная реализация SQL-92.
	      \end{itemize}
	\item Apache Derby
	      \begin{itemize}
		      \item In-memory mode -- умеет быть полностью in-memory, а также умеет работать с данными которые в память
		            не поместились.
		      \item Хорошо совместим с DB2, так как был проектом IBM'а, и из-за этого же не очень хорошо совместим со
		            всеми остальными.
		      \item Pure Java -- встраивается в любое Java приложение.
	      \end{itemize}
	\item HyperSQLDB
	      \begin{itemize}
		      \item Pure Java.
		      \item Не поддерживает транзакции.
		      \item In-memory mode.
		      \item В основном используется для тестирования.
	      \end{itemize}
	\item Access
	      \begin{itemize}
		      \item Совмещение СУБД и RAD.
		      \item Встраиваемые приложения.
	      \end{itemize}
	      \begin{remark}
		      In-memory базы данных хорошо подходят для тестирования, потому что каждый пользователь может легко
		      поднять свой instance из-за того что база in-memory и это всё ещё SQL, и каждому из instance'ов не
		      будут мешать тесты других пользователей, также нет проблем с тем что схема данных может быть
		      старой.
	      \end{remark}
\end{itemize}

