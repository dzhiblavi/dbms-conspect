\subsection{Современные РСУБД}
\subsubsection{Корпоративные}

\paragraph{Корпоративные СУБД} - предназначены для продажи большим корпорациям, но у большинства таких РСУБД есть
девелоперские лицензии, которые позволяют использовать их в ограниченной среде (ограничение на кол-во ядер и размер памяти).

\begin{itemize}
    \item Oracle (Oracle)
    \begin{itemize}
        \item высокая пропускная способность (throughput)
        \item невысокая скорость обновления (latency)
    \end{itemize}
    \begin{remark}
         Два утверждения выше не противоречат друг другу, хотя каждый запрос и исполняется медленно, пропускная способность получается большой за счёт того, что конкретная СУБД заточена на исполнение тысяч параллельных запросов, и суммарная пропускная способность всех этих запросов, а не отдельных запросов, будет высокой.
    \end{remark}
    \item DB2 (IBM)
    \begin{itemize}
        \item ориентация на «большие» машины, то есть с точки зрения IBM, СУБД это не приложение, которое крутится на сервере, а отдельное железо
        \item мало распространена в России, так как развивалась в 60'е - 80'е годы предыдущего века
        \item неполная совместимость с SQL
    \end{itemize}
    \item SQL Server (Microsoft)
    \begin{itemize}
        \item работа под Windows
        \item масштабируемость (путём добавления новых процессоров)
    \end{itemize}
\end{itemize}

\subsubsection{Свободные}
\begin{itemize}
    \item MySQL
    \begin{itemize}
        \item поддерживаются различные форматы хранения БД
        \item неполная поддержка SQL
        \item есть enterprise и community версии
    \end{itemize}
    \item PostgreSQL
    \begin{itemize}
        \item непосредственная поддержка связей - СУБД достаточно стабильна, чтобы использовать в реальных проектах
        \item объектные расширения - но в то же время эта СУБД - экспериментальный проект, в который добавляется куча различных возможностей, некоторые из которых не выходят из экспериментального статуса
    \end{itemize}
    \item Firebird
    \begin{itemize}
        \item была очень популярна когда делалась \href{https://en.wikipedia.org/wiki/Borland}{Borland}'ом под Delphi
        \item используется только в старых проектах, так как в БД, которые используют это СУБД, есть данные, которые нельзя потерять, а перенести их очень сложно
    \end{itemize}
\end{itemize}

\subsubsection{Встраиваемые}
\begin{itemize}
    \item SQLite
    \begin{itemize}
        \item компактна, поэтому много используется на мобильных устройствах
        \item in-memory mode - все данные должны поместиться в память
        \item ограниченная реализация SQL-92
    \end{itemize}
    \item Apache Derby
    \begin{itemize}
        \item in-memory mode - умеет быть полностью in-memory, а также умеет работать с данными которые в память не поместились
        \item хорошо совместим с DB2, так как был проектом IBM'а, и из-за этого же не очень хорошо совместим со всеми остальными
        \item pure Java - встраивается в любое java приложение
    \end{itemize}
    \item HyperSQLDB
    \begin{itemize}
        \item pure Java
        \item не поддерживает транзакции
        \item in-memory mode
        \item в основном используется для тестирования
    \end{itemize}
    \item Access
    \begin{itemize}
        \item совмещение СУБД и RAD
        \item встраиваемые приложения
    \end{itemize}
    \begin{remark}
        In-memory базы данных хорошо подходят для тестирования, потому что каждый пользователь может легко поднять свой instance из-за того что база in-memory и это всё ещё SQL, и каждому из instance'ов не будут мешать тесты других пользователей, также нет проблем с тем что схема данных может быть старой.
    \end{remark}
\end{itemize}