\subsection{Оптимизация запросов. Оценка размера и распределения}

\subsubsection{Статистические характеристики}

Рассмотрим оценки трудозатрат, полезные с точки зрения оптимизации запросов:

\begin{itemize}
	\item \textbf{Получение всех результатов}. Для оценки трудоемкости обычного запроса;
	\item \textbf{Получение первого результата}. Для оценки трудоемкости результата
	      \texttt{exists};
	\item \textbf{Получение части результатов}. Для оценки трудоемкости поддерживаемого во многих
	      СУБД получения данных с \texttt{offset} и \texttt{limit}.
\end{itemize}

Для оптимизации запросов оцениваются следующие параметры результатов запросов:

\begin{itemize}
	\item \textbf{Оценка размера}. Позволяет оценить объем данных, и в частности, поместится ли
	      результат в оперативную память. При этом учитывается, что ложноположительный ответ может привести
	      существенным спадам производительности относительно оценки;
	\item \textbf{Оценка распределения}. Позволяет уточнить оценки размера и оптимизировать более
	      сконцентрированные участки данных, например, для соединений.
\end{itemize}

\paragraph{Статистические характеристики}

\begin{itemize}
	\item \textit{Основные статистики}: среднее, медиана, мода;
	\item \textit{Характеристики распределения}: минимум, максимум, дисперсия;
	\item \textit{Размеры выборки}: число значений, число различных значений;
	\item \textit{Вид распределения}: распределение по частотам.
\end{itemize}

\paragraph{Виды распределений}

\begin{itemize}
	\item \textit{Распределение отношения}: построчные, постраничные (лучше отражает реальные
	      трудозатраты, поскольку чтение идет страницами);
	\item \textit{Распределение атрибутов}: значения, размер, диапазон, распределение;
	\item \textit{Распределение индексов}:
	      \begin{itemize}
		      \item \textit{Индексы на основе деревьев}: число листов, промежуточных листов, высота
		            дерева;
		      \item \textit{Хеш индексы}: размеры корзин, число уникальных значений;
	      \end{itemize}
\end{itemize}

\paragraph{Упрощающие предположения}

Гипотезы о данных, которые могут упрощать анализ:

\begin{itemize}
	\item \textit{Равномерность распределений}. Количества кортежей с равными значениями примерно
	      равны между собой;
	\item \textit{Независимость атрибутов}. Распределения независимы. В реальном мире взаимосвязь
	      может быть существенной;
	\item \textit{Равномерность запросов}. Запросы ко всем значениям равновероятны. В реальном мире
	      может нарушаться в зависимости от времени (к старым данным обращение происходит реже);
	\item \textit{Равномерность заполнения}. Страницы содержат равное число кортежей;
	\item \textit{Случайность расположения}. Кортежи распределены по страницам случайным образом;
\end{itemize}

\begin{theorem}[Христодоулакиса]
	Применение предположений дает оценки для худшего случая.
\end{theorem}

\paragraph{Оцениваемые значения}

\begin{itemize}
	\item \textbf{Математическое ожидание числа чтений страниц}. Обработка результатов чтения
	      требует существенно меньше ресурсов, чем чтение.
	\item \textbf{Математическое ожидание числа различных значений}. Помогает оценить результат
	      фильтрации.
\end{itemize}

\subsubsection{Оценка операций}

\textit{Общий механизм} -- по базовым отношениям собирается статистика, в том числе взятая из
индекса, а также результаты операции оцениваются исходя из распределения аргументов и свойств
операций.

\paragraph{Оценка фильтрации}

Введем оценочную функцию:

\begin{align}
	|Out| = Sel(expr)|In|
\end{align}

\begin{definition}
	\textit{Селективность ($Sel(expr)$)} -- отношение размера результата к размеру входа.
\end{definition}

СУБД может производить следующие сравнения для оценки эффективности:

\begin{itemize}
	\item \textit{Селективность из распределения},
	\item \textit{Интервальные оценки},
	\item \textit{Сравнение с другим атрибутом}.
\end{itemize}

Оценка селективности составных выражений:

\begin{align}
	Sel(A \wedge B) & = Sel(A) Sel(B | A) \approx Sel(A) Sel(B) \\
	Sel(A \vee B)   & = Sel(A) + Sel(B) - Sel(A \wedge B)
\end{align}

\paragraph{Оценка соединения}

Введем оценочную функцию:

\begin{align}
	|Out| = Sel(expr)|R_1||R_2|
\end{align}

При этом планировщик может учитывать, какие имеются ключи для соединяемых отношений. В частности,
если соединение проходит по внешнему ключу, ссылающемуся на ключ другой таблицы, то каждому кортежу
левого аргумента будет соответствовать не более одного кортежа правого аргумента, то есть:

\begin{align}
	|Out| = \min(|R_1|, |R_2|)
\end{align}

\paragraph{Множественные операции}

\begin{itemize}
	\item \textbf{Пересечение}.
	      \begin{align}
		      0 \leq |Out| \leq \min(|R_1|, |R_2|)
	      \end{align}
	\item \textbf{Объединение}.
	      \begin{align}
		      \max(|R_1|, |R_2|) \leq |Out| \leq |R_1| + |R_2|
	      \end{align}
	\item \textbf{Разность}.
	      \begin{align}
		      \max(|R_1| - |R_2|, 0) \leq |Out| \leq |R_1|
	      \end{align}
\end{itemize}

\paragraph{Сбор статистики}

Для пересчета статистики во многих СУБД поддерживается оператор \texttt{analyze}, который не
входит в стандарт SQL. Использование целесообразно после массовых обновлений или периодически.

Гистограммы распределений пересчитываются по индексам или при явном пересчете.

Также допустимо делать предположения о распределениях и проверять гипотезы статистическими тестами:
на равномерность и нормальность.
